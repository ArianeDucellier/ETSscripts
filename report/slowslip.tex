\documentclass[main.tex]{subfiles}
 
\begin{document}

\part{Slow slip}

\chapter{Data}

The data are available on the website of the Pacific Northwest Geodetic Array (\href{http://www.geodesy.cwu.edu/}{PANGA}), from Central Washington University. Three types of time series are available:

\begin{itemize}
\item Raw data recorded by the GPS stations, after GIPSY postprocessing.
\item Detrended data, for which a linear trend corresponding to the secular plate motion has been removed from the data.
\item Cleaned data, for which the linear trend, steps due to earthquakes or hardware upgrades, and annual and semi-annual sinusoids signals have been simultaneously estimated and removed following Szeliga \textit{et al.} (2008 ~\cite{SZE_2008}).
\end{itemize}

For each GPS station, the website provides a file for the three components of the displacement (latitude, longitude, and vertical), and each file contains three columns, corresponding to the time, the displacement (in millimetres), and the error. The data are recorded once a day. \\

The data are filtered with the function $f (t)$:

\begin{equation}
f (t) = \textrm{line} (t) + \textrm{annual} (t) + \textrm{jumps} (t)
\end{equation}

with:

\begin{equation}
\textrm{line} (t) = p_1 + p_2 t
\end{equation}

\begin{equation}
\textrm{annual} (t) = p_3 \sin (2 \pi t + p_4)
\end{equation}

\begin{equation}
\textrm{jumps} (t) = \sum_{i = 1}^{n} p_i \textrm{Heaviside} (t - t_i)
\end{equation}

The values of the $p_i$ and $t_i$ are given at the beginning of each file.

Add something about the inversion of these parameters (Nikolaidis, 2002), and about the seasonal trends (Blewitt and Lavall\'ee, 2002)). 

\chapter{Method}

\chapter{Results}

\end{document}
