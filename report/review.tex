\documentclass[main.tex]{subfiles}
 
\begin{document}

\part{Literature review}

Dragert \textit{et al.} (2001 ~\cite{DRA_2001}) carefully analyzed data from 14 GPS sites in Cascadia. They detected a reversed direction of motion, lasting for a few days, at several sites located landward from the locked seismogenic zone. The observed signal propagated parallel to the strike of the subducting slab at around 6 km per day. They fitted the observed displacements with the displacements produced by a fault located at the plate boundary in an elastic half-space, and concluded that the slip must have occurred downdip of the locked and transition zones. The total moment derived from their numerical modelling was equivalent to a M\textsubscript{w} = 6.7 earthquake. They suggested that deep-slip events like this could play a key role in the stress loading of the seismogenic zone, and therefore trigger megathrust earthquakes. \\

Obara (2002 ~\cite{OBA_2002}) computed the cross correlations of envelope seismograms between pairs of stations of the Hi-net network in southwestern Japan, while moving the time lag between the two traces. It allowed him to identify long coherent signals, with a predominant frequency of 1 to 10 Hz. A rough estimate of the propagation velocity of these tremors led him to conclude that they were propagated by S-wave velocity. The tremors were located in the western part of Shikoku and in the Kii peninsula, near the Mohorovi\u ci\'c discontinuity. Obara speculated that the observed tremors may be a continuous sequence of low frequency earthquakes. Due to the long duration of the tremors and the mobility of the tremor activity, he suggested that the occurrence of tremors may be related to the release of fluids in the subduction zone. \\

Preston \textit{et al.} (2003 ~\cite{PRE_2003}) analyzed first and second arrivals travel times from marine active sources, and first arrival travel times from local earthquakes in Cascadia. They inverted their data for 3D P-wave velocity structure, earthquake locations, and geometry of the reflector. They interpreted the reflector as the Moho of the subducting Juan de Fuca plate. The earthquakes were separated into two groups. The earthquakes in the first group were located in the oceanic mantle, up-dip of the Moho's 45km depth contour, and might be caused by serpentinite dehydration. The earthquakes in the second group occurred in the subducted oceanic crust, down-dip of the Moho's 45km depth contour, and might be caused by basalt-to-eclogite transformation. No other reflector that could correspond to the plate boundary and the upper limit of a low velocity zone was detected. \\

Rogers and Dragert (2003 ~\cite{ROG_2003}) examined a number of seismograph signals in Vancouver Island together and observed tremorlike signals identified by the similarities in their envelopes. These signals were characterized by frequencies between 1 and 5 Hz, pulses of energy, and a duration from a few minutes to several days. Their occurrence was correlated temporally and spatially with slip events on the subduction zone interface. The source depth of the tremors was comprised between 25 and 45 km, which is compatible with a location on the plate interface where the slip was supposed to occur. They named this phenomenon Episodic Tremor and Slip (ETS). They believed that the cause of these tremors could be a shearing source and that fluids may play an important role in their generation. \\

Obara \textit{et al.} (2004 ~\cite{OBA_2004}) observed tiltmeters data in boreholes in western Shikoku, and compared them with tremor activity. They noticed the simultaneous occurrence and migration of slow slip and tremors, with a recurrence interval of approximately six months. They divided the phenomena into two steps, modelled the slow slip by a dislocation on two reverse faults, and found a good agreement between predicted slip and tilt observations. The updip limit of the slip corresponded to the location of the tremors, whereas the downdip limit of the slip corresponded to the junction between the top of the subducting oceanic plate and the continental Moho. They concluded that the coupling phenomenon producing both tremors and slow slip should be located on the plate boundary  and be related to the stress accumulation of the locked zone.\\

Szeliga \textit{et al.} (2004 ~\cite{SZE_2004}) used a combination  of data from GPS and seismic stations to look for slow slip events in southern Cascadia. The GPS time series from the Pacific Northwest Geodetic Array and the Bay Area Regional Deformation Array were stabilized using a reference set of stations from stable North America. The final time series were then detrended, and offsets due to hardware upgrades, earthquakes, and atmospheric seasonal effects, were removed. Recordings from seismic stations of the Northern California Seismic Network (NCSN) were used to count the number of hours of tremor per week. The authors observed a correlation between the tremor rate and deformation reversals. Slow slip events occur about every 11 months in northern California, and sporadic events were observed in central Oregon. Slow slip events may thus not be confined to the Puget Sound area, but may occur throughout Cascadia. \\

S\'anchez-Sesma and Campillo (2006, ~\cite{SAN_2006}) studied the relationship between the Green's function and the cross correlation in an elastic isotropic medium. They assumed an equipartition of elastic plane waves, that is the energy is equally distributed among all possible modes and all possible direction of propagation. The wave field is thus diffuse, and the ratio between The \textit{P} and the \textit{S} spatial energy densities is equal to $E_S / E_P = (\alpha / \beta)^2$ in 2D and $E_S / E_P = 2 (\alpha / \beta)^3$ in 3D. They considered the case of an \textit{SH} wave in a 2D medium, the case of \textit{P} and \textit{SV} waves in a 2D medium, and the case of the 3D medium. For each case, they computed on one hand the cross correlation of the displacement at a point $x$ and the displacement at a point $y$, and on the other hand the Green's function between point $x$ and point $y$. Using the value of the \textit{P} to \textit{S} energy ratio, they established that in all three cases the average cross correlation over all directions of motion between two points is proportional to the imaginary part of the Green's function between these two points. \\

Shelly \textit{et al.} (2006 ~\cite{SHE_2006}) used a combination of waveform cross-correlation and double-difference tomography to get a precise location of the source of low-frequency earthquakes (LFE), and a high-resolution velocity structure, in western Shikoku. They observed that the LFEs are located on a plane 5-8 km above the dipping plane of regular seismicity within the subducting plate, and concluded that the LFEs must occur at the plate interface, while regular seismicity begins within the lower part of the crust at shallower depths, and expand upwards the crust as the slab subducts. They also found a zone of high $V_P / V_S$ ratio in the vicinity of the LFEs, which they interpret as high pore-fluid pressure. They hypothesize that the LFEs may be generated by local shear slip accelerations due to local heterogeneities during large slow slip events at the plate interface, and that the high pore-fluid pressure might enable slip by reducing normal effective stress. \\

Ide \textit{et al.} (2007 ~\cite{IDE_2007_GRL}) computed cross correlation of low-frequency earthquakes (LFE) waveforms in western Shikoku, and concluded from their predominantly positive distribution that mechanisms of LFEs are very similar. Then, they compared the polarity of regular intraplate earthquake waveforms and stacked LFE waveforms in order to identify the nodal planes of P-wave radiation. Finally, they inverted the moment tensor of a reference LFE using S-waveforms and intraplate earthquakes waveforms as empirical Green's function. The focal mechanism was a shallow dipping thrust fault, consistent with the subduction of the Philippine plate. The fact that this mechanism is consistent with fault models for slow slip events led them to conclude that slow slip, LFEs and tremors must be different manifestation of the same process. \\

Ide \textit{et al.} (2007 ~\cite{IDE_2007_nature}) studied the scaling and spectral behaviour of slow earthquakes (e.g. silent earthquakes, low-frequency earthquakes, low-frequency tremor, slow slip events, and very-low-frequency earthquakes). They observed that their seismic moment is proportional to their duration (and not the cube of the duration, as is the case for regular earthquakes), and that their seismic moment rate is proportional to the inverse of the frequency (and not the inverse of the square of the frequency, as is the case for regular earthquake). They proposed two models to explain this behaviour. In the constant low-stress drop model, the propagation velocity is proportional to the square of the characteristic length. In the diffusional model, the slip is constant and the propagation velocity is proportional of the characteristic length. They conclude that all slow earthquakes are different manifestations of the same physical phenomenon, and constitute a new earthquake category. \\

Shelly \textit{et al.} (2007 ~\cite{SHE_2007_nature}) observed that tremor and low-frequency earthquakes (LFE) had similar frequency content, distinct of the spectra of background noise and regular earthquakes. They used well recorded LFEs in Shikoku as template events, and cross correlated their waveforms with tremor waveforms recorded in the same area. They stacked the corresponding correlation coefficients for the three components of several stations, and identify the tremor waveform as a small LFE if the summed correlation coefficient was higher than a threshold. They found that the events thus detected had a good spatial coherence, and that most of the tremor signal could be explained by a swarm of small LFEs. They concluded that the tremors were generated by the same mechanism that causes LFEs and slow slip, that is fluid enabled shear slip on the plate boundary. \\

Szeliga \textit{et al.} (2008 ~\cite{SZE_2008}) determined the timing and the amplitude of 34 slow slip events throughout the Cascadia subduction zone between 1997 and 2005. They stabilized the GPS time series using a reference set of stations from stable North America. They then modelled the GPS time series by the sum of a linear trend, annual and biannual sinusoids representing seasonal effects (Blewitt and Lavall\'ee, 2002), and Heaviside step functions corresponding to earthquakes and hardware upgrades. The linear system was then solved using a weighted QR decomposition (Nikolaidis, 2002). Finally, they applied a Gaussian wavelet transform to the residual time series to get the exact timing of the slow slip at each GPS station. The succeeding wavelet basis functions are increasingly sensitive to temporal localization of a given signal, and the onset of faulting appears on the wavelet spectrum as an amplitude spike present over all frequencies. The offset for each slow slip event was then used to invert for the slow slip at depth by assuming a thrust fault slip at each subfault of the plate boundary. An equivalent moment magnitude is thus obtained. The authors noted that the events are disconnected, and short-lived (1 to 7 weeks with a maximum displacement of 1 centimetre), and that they do not extend to the tightly coupled shallower plate interface. No long term event was observed during the time period studied. Finally, the lack of resolvable vertical deformation prevented the authors from giving a constraint on the downdip extent of the slow slip. \\

Wech and Creager (2008 ~\cite{WEC_2008}) computed the cross correlations of envelope seismograms for a set of 20 stations in western Washington and southern Vancouver Island. They performed a grid search over all possible source locations to determine which one minimizes the difference between the maximum cross correlation and the value of the correlogram at the lag time corresponding to the S-wave travel time between two stations. This location method is automated (and thus, less labor intensive), makes use of near real-time regional data, and is less computationally intensive than previously proposed methods. They applied their method to the 2007 and 2008 Episodic Tremor and Slip (ETS) events, and located the epicentres in the region where the plate interface is 30-45 km deep and found a sharp updip boundary of the tremor 75km east of the downdip edge of the megathrust zone. Moreover, they identified between the two ETS events geodetically undetectable tremor that represents nearly half of the total amount of tremor. They concluded that their location method could help mapping the transition and locked zones of the plate boundary. \\

Audet \textit{et al.} (2009 ~\cite{AUD_2009}) computed receiver functions of teleseismic waves in Vancouver Island, and analyzed the delay times between the forward-scattered P-to-S, and back-scattered P-to-S and S-to-S conversions at two seismic reflectors identified as the top and bottom of the oceanic crust. It allowed them to compute the P-to-S velocity ratio (V\textsubscript{P}/V\textsubscript{S}) of the layer and the S-wave velocity contrast at both interfaces. The very low Poisson's ratio of the layer could not be explained by the composition, and they interpreted it as evidence for high pore-fluid pressure. They explained the sharp velocity contrast on top of the layer as a low permeability boundary between the oceanic plate and the overriding continental crust. They concluded that the high pore-fluid pressures in the oceanic crust could explain the recurrence of Episodic Tremor and Slip (ETS) events, either by hydrofracturing, either by extending the region of conditionally stable slip.\\

La Rocca \textit{et al.} (2009 ~\cite{LAR_2009}) stacked seismograms over all stations of the array for each component, and for three arrays in Cascadia. They then computed the cross-correlation between the horizontal and the vertical component, and found a distinct and persistent peak at a positive lag time, corresponding to the time between P-wave and S-wave arrivals. Using a standard layered Earth model, and horizontal slowness estimated from array analysis, they computed the depths of the tremor sources. They located the sources near or at the plate interface, with a much better depth resolution than previous methods based on seismic signal envelopes, source scanning algorithm, or small-aperture arrays. They concluded that at least some of the tremor consisted in the repetition of low-frequency events as was the case in Shikoku. A drawback of the method was that it could be applied only to tremor located beneath an array, and coming from only one place for an extended period of time. \\

Ghosh \textit{et al.} (2010 ~\cite{GHO_2010_G3}) used a beam backprojection (BBP) method to detect and locate tremor from seismic recordings of a small-aperture array in the Olympic Peninsula. They observed tremor propagating near-continuously in the slip-parallel direction, at velocities between 30 and 200 km/h, and for distances up to 40 km. They proposed two mechanisms to explain these tremor streaks. In the first model, the slip and the plate dip direction differ by up to $\theta$ = 35\degree. The tremor propagates along slip-parallel striations, corrugations, and ridge-and-groove structures on the fault surface. The long-term front velocity $V_L$ and the short-term streak velocity $V_S$ are related by $\sin \theta = \frac{V_L}{V_S}$. In the second model, periodic breaking of the impermeable caprock increases the pore pressure and creates a pressure gradient that will in turn induce fluid flow along a conduit made available by striations and grooves on the fault plane. However, this last model requires long continuous conduits that seem unlikely. Tremor distribution thus varies over different time scales: along-strike migration of the front at about 10 km/day, rapid tremor reversals at about 10 km/h, and along-slip tremor streaks at about 30-200 km/h. Moreover, the moment release of tremors is distributed among patches. \\

Ohtani \textit{et al.} (2010 ~\cite{OHT_2010}) designed the Network Stain Filter (NSF) to detect transient deformation signals from large-scale geodetic arrays. Contrary to their previous work on the Network Inversion Filter (NIF), there is no need to specify potential sources of deformation. They modelled the position of the GPS station by the sum of the secular velocity, a spatially coherent field, site-specific noise, reference frame errors, and observation errors. The spatial displacement field is modelled by the sum of basis wavelets (the Deslauriers-Dubuc wavelet of degree 3) with time-varying weights. The transient is considered to be nearly steady-sate, so that it has spatial weights for the displacement and the velocity, but the acceleration is modelled by a random walk with a time-varying variance. All the time varying coefficients are estimated using Kalman filtering, and the optimization problem is regularized with the spatial sum of the transient strain rate field. The authors first applied the NSF to a synthetic dataset and obtained a good recovery of the transient signal although its amplitude was similar to the noise level. They then applied the NSF to the 1996 Boso slow earthquake, and could detect the transient, but with an oversmoothing in the time domain, and could not reproduced the abrupt time history. They concluded that the NSF could be used to detect coherent motions that could be good candidates for further analysis, but that the ovsersmoothing problem should be addressed. \\

Alba (2011 ~\cite{ALB_2011}) used hourly water level records from four tide gauges in the Juan de Fuca Straight and the Puget Sound to determine vertical displacements, uplift rates between Episodic Tremor and Slip (ETS) events, and net uplift rates between 1996 and 2011. The noise in the tide gauges data is associated with tides, and ocean and atmospheric noise on multiple timescales (a few days for storms to decades for oscillations between ocean basins), and is assumed to be coherent between each of the four tidal gauges studied. On the contrary, the uplift due to ETS events should be different at each tidal gauge. The author first removed the tides using NOAA hourly harmonic tidal predictions. She then removed the residual noise using two different methods. The first method is based on the Discrete Wavelet Transform (DWT). More precisely, the author applied a DWT to each of the four sites studied, and to the average of the four sites. Then, for each level of the DWT decomposition, she carried out a linear regression between the detail for one site and the detail for the average of the four sites. This process gives a coefficient for each level and for each site. She then constructed a noise signal for each site by multiplying the coefficient from the linear regression by the detail of the average over the four sites, and summing for all levels. The noise signal thus obtained was then removed from the time series. The second method uses a frequency domain transfer function to remove coherent noise at certain frequencies. She then stacked multiple events to obtain an average event uplift rate, aligning the 12 ETS events using exact timing from GPS data. A difference in uplift between the two tidal gauges at Port Angeles and Port Townsend was then clearly seen in the stacked time series. Finally, the author removed the long-term uplift rate and the long-term sea level rise to obtain an average inter-event uplift rate. She found that the inter-event deformation at a site is equal and opposite to the deformation during an ETS event, suggesting that ETS events are, on average, releasing the strain accumulated between ETS events. \\

Wech and Creager (2011~ \cite{WEC_2011}) studied the variations of slip size and periodicity of slow slip with increasing depth in the Cascadia subduction zone. They used the waveform envelope correlation and clustering (WECC) method developed in their previous work (Wech and Creager, 2008 \cite{WEC_2008}) to detect and locate tremor epicentres, and assumed that slow slip happens at the same time and location as tremor. They then divided the tremor region into four 20-km-wide strike-perpendicular bins, and found evidence of small and frequent slip on the downdip size of the tremor zone, and larger and less frequent slip on the updip size. They speculate that higher temperatures at higher depths would produce lower frictional strength and a weaker fault. Each small slip event would thus transfer stress updip to a stronger portion of the fault, with a higher stress threshold. When enough stress has been transferred, this updip portion would slip and transfer slip further updip of the fault. \\

Bostock \textit{et al.} (2012 ~\cite{BOS_2012}) looked for low-frequency earthquakes (LFE) by computing autocorrelations of 6-second long windows for each component of 7 stations in Vancouver Island. They then classified their LFE detections into 140 families. By stacking all waveforms of a given family, they obtained an LFE template for each family. They extended their templates by adding more stations and computing cross correlations between station data and template waveforms. They used P- and S-traveltime picks to obtain an hypocentre for each LFE template and concluded that the LFEs were located on the plate boundary and that their downdip extension coincided with the seaward extrapolation of the continental Moho. By observing the polarizations of the P- and S-waveforms of the LFE templates, they computed focal mechanisms and obtained a mixture of strike slip and thrust mechanisms, corresponding to a compressive stress field consistent with thrust faulting parallel to the plate interface. \\

Ghosh \textit{et al.} (2012 ~\cite{GHO_2012}) used multibeam-backprojection (MBBP) to detect and locate tremor with much higher resolution. They used data recorded by 8 small-aperture seismic arrays in the Olympic peninsula during the large August 2010 Episodic Tremor and Slip (ETS) event and an entire inter-ETS cycle. They observed that the tremors were located near the plate boundary, on a layer parallel to and a few kilometres above the layer of regular earthquakes. Distinct patches, tens of kilometres of dimension, were found to produce the majority of the tremor. The propagation velocity varied from 4 to 20 km/day at large time scale (days), and up to 100 km/h at small time scale (minutes). They interpreted their observations with a model made of patches of asperities surrounded by regions slipping aseismically. Propagation velocity was supposed to be slow in the asperities area, and fast outside of the asperities area. \\

Wei \textit{et al.} (2012 ~\cite{WEI_2012}) used the Network Strain Filter (NSF) to analyze data from 54 continuous GPS stations in the Alaska subduction zone from 2007 to 2011. They assumed that the displacement observed at a GPS station is the sum of the secular velocity, a transient field, site-specific noise, seasonal variations, reference frame errors, and observation errors. They observed an increase in the velocity rate starting at the beginning of 2010. They modelled the transient displacement field by the slip on a planar fault in an elastic homogeneous half-space, and concluded that the observations are consistent with an accumulated slip of magnitude $Mw$ 6.9 in 23 months. They modelled the slab seismicity with an Epidemic Type Aftershock Sequence (ETAS) model, and observed an increase in the background seismicity rate beginning in mid-2010. However, no increase in the number of non volcanic tremor was reported in the area. They observed that there is a lack of large aftershock sequences in the depth range of the slab (30-70 km), which is an indicator of very low effective normal stress, and could be linked to the occurrence of long-duration slow slip events. \\

Bostock (2013 ~\cite{BOS_2013}) proposed a new model to explain the nature of a landward-dipping, low velocity zone (LVZ) that was detected in most subduction zones. Previous models in Cascadia interpreted the LVZ as the entire oceanic crust, an extended plate boundary, serpentinized material above the plate boundary, or a fluid-rich layer in the overriding continental crust. In the new model, the LVZ is interpreted as upper oceanic crust. The upper oceanic crust is hydrated by hydrothermal circulation at the ridge. The free water is the incorporated into hydrous minerals. As subduction begins, prograde metamorphic reactions release hydrous fluid in the upper oceanic crust. They stayed trapped by an impermeable upper plate boundary and the impermeable gabbroic lower oceanic crust. The high pore-fluid pressure explains the low shear wave velocity and the high Poisson's ratio. At about 45km depth, the onset of eclogitization liberates additional fluids and causes volumetric changes that break the plate boundary seal. The penetration of hydrous fluids in the mantle wedge leads to serpentinization of the mantle wedge material and erasure of the Moho's seismic contrast. By 100 km depth, the eclogitization is largely completed and the LVZ disappears. \\

Nowack and Bostock (2013 ~\cite{NOW_2013}) used a set of 140 low-frequency earthquakes (LFE) waveform templates in southern Vancouver Island as a record of empirical Green's functions. They used a regional 3D tomographic model, and inserted a low velocity zone under the plate boundary. They computed synthetic waveforms of a pulse using 3D ray-tracing for different source locations corresponding to the locations of the LFE waveforms templates. They then compared their synthetics to the data from the LFE templates, and carried out a grid search to check which values of P-wave velocity, ratio of P- and S-wave velocities and thickness of the low velocity zone gave the better fit. Their estimates of the thickness of the low velocity zone, the velocity contrast and the ratio between P- and S-wave velocities were consistent with the results from previous teleseismic studies. \\

Armbruster \textit{et al.} (2014 ~\cite{ARM_2014}) proposed a new method to accurately locate tremor sources. They started with seismic data from two stations and computed the cross correlation of the seismic signals on 150 seconds time windows. They carried out a grid search on the polarization angles of each station and the offset time between both stations, and look for the greatest cross correlation value. They assumed that a tremor event occurred when the polarization angles and the offset times are consistent for several consecutive time windows. They extended the method for three stations, and looked for the consistency between the polarization angles and offsets found for the three possible pairs of stations, without imposing a duration criterion. Finally, they used the waveforms containing tremors from the three-stations detections to look for S-wave and P-wave at additional stations. With four S-wave detections and one P-wave detection, they were able to retrieve the location and depth of the tremor source with a 1 to 2 kilometres accuracy. They noticed that the polarization of the waveforms were consistent with a shear mechanism on the plate boundary. They also found out a similarity of pattern of the locations of the tremor sources for the three main Episodic Tremor and Slip (ETS) events for which they analyzed seismic recordings. \\

Audet and B\"urgmann (2014 ~\cite{AUD_2014}) studied the relationship between the ratio between P-wave velocity and S-wave velocity in the subducted oceanic crust and the forearc and the periodicity of slow earthquakes. They computed the $V_P / V_S$ ratio from receiver functions and data from the literature. They noticed that slow earthquakes are associated with a high $V_P / V_S$ ratio in the subducted oceanic crust, but without relationship with recurrence time. However, they pointed out that the recurrence time of slow earthquakes increases linearly with the $V_P / V_S$ ratio of the forearc. Moreover, along a margin-perpendicular profile from northern Cascadia, the $V_P / V_S$ ratio of the forearc, and the recurrence time of Episodic Tremor and Slip (ETS) events, decrease with increasing depth. The authors explained the low $V_P / V_S$ ratio in the forearc by the enrichment of forearc minerals in fluid-dissolved silica derived from the dehydration of the downgoing slab. However, they estimated that the fluid flux required for the formation of quartz veins was two orders of magnitude greater than the fluid production rates estimated from the dehydration of the slab.They hypothesized that silica-saturated fluids may originate from the complete serpentinization of the mantle near the wedge corner. They suggested that higher temperature and quartz content at depth may lead to faster dissolution - precipitation processes and more frequent slip events. Their model could also explain the global variation in recurrence time, with mafic silica-poor regions having longer ETS recurrence times that felsic silica-rich regions. \\

Problem: What controls the recurrence interval of slow earthquakes?

Before: Association of slow earthquakes with a dipping layer of low seismic velocity. Segmentation of ETS recurrence in Cascadia correlates with overriding forearc structure and geology.

Method: Receiver functions + two-layer model (low velocity zone and forearc)  $\rightarrow$ Vp / Vs ratio for the two layers. Recurrence time of slow earthquakes from the literature.

After: Recurrence time increases linearly with Vp / Vs ratio of forearc. Vp / Vs ratio of forearc (and recurrence time) decreases with increasing depth. Low Vp / Vs ratio can be explained by silica-rich minerals in the forearc. Possible silica enrichment from fluid-dissolved silica derived from dehydration of downgoing slab.

Open questions: Estimated fluid flux required for quartz-vein formation is two orders of magnitude greater than fluid production rates from slab dehydration $\rightarrow$ Fluids may come from serpentinization of the mantle wedge corner. Higher T and quartz content $\rightarrow$ Faster dissolution - precipitation processes $\rightarrow$ More frequent slip events. Explain global variation in recurrence time: mafic silica-poor regions have longer recurrence times that felsic silica-rich regions.

Idehara \textit{et al.} (2014 ~\cite{IDE_2014}) studied the temporal clustering of tremor activity in major tectonic zones worldwide. They defined a tremor event as a period with continuing recorded tremor activity from a source located within the same bin of radius about 10-12 kilometres. The event duration is the half-width of the stacked envelope of the seismic waveforms for many stations. They analyzed the frequency distribution of the waiting time between two tremor events, and found a bimodal distribution. They then computed the correlation integral between event times. For waiting times smaller than a characteristic time $\tau_c$, the correlation integral is non-Poissonian and seems to follow a power law. For waiting times larger than $\tau_c$, the correlation integral follows a Poisson distribution. They applied a $\chi^2$ test to verify when the correlation integral was statistically significantly different from a Poisson distribution, and defined $\tau_c$ as the longest waiting time for which the difference is statistically significant. The authors then computed the values of $\tau_c$ for each bin in different tectonic regions worldwide. They found that along dip, $\tau_c$ is decreasing with increasing depth. In Shikoku and Kii-Tokai, the along-strike heterogeneities of $\tau_c$ seem to correlate with localized seismic velocity anomalies. Moreover, there is a small correlation between $\tau_c$ and tremor duration. The authors interpreted the along-dip variations in $\tau_c$ to variations in fault strength due to thermal conditions, or to stress transfer along the plate interface. They hypothesized that along-strike variations may be due to other factors such as pore fluid pressure or the geometry of the plate interface, and that regional variations may be due to variable maturity of the plate interface. \\

Royer and Bostock (2014 ~\cite{ROY_2014}) generated low-frequency earthquake (LFE) templates in northern Cascadia using the same processing steps (network autocorrelation, waveform correlation cluster analysis and network cross correlation) as in Bostock \textit{et al.} (2012 ~\cite{BOS_2012}). They identified their LFE templates as empirical Green's functions, which justifies their subsequent use in waveform inversions. They computed template locations using standard linearized inversion and double difference algorithm, and concluded that LFE templates parallel the plate boundary. They carried out a moment tensor inversion for each LFE template and found out that a majority of the focal mechanisms were consistent with shallow thrust faulting, although there is more variability in northern Washington state due to poorer station coverage and lower signal-to-noise ratio. \\

Thurber \textit{et al.} (2014 ~\cite{THU_2014}) compared the efficiency of linear and phase-weighted stacking for picking low-frequency earthquakes (LFEs) arrivals. Once initial templates have been identified using the cross-station method of Savard and Bostock (2013), the signal is stacked using linear or phase-weighted stacking. The author then used an iterative procedure in which, at each iteration, they cross-correlate the stack with the continuous seismic signal, detecting new LFEs. At the end of each iteration, all the LFE waveforms are stacked to produce a new template with a higher SNR. The phase-weighted stack produced faster a little more detections than the linear stack, and a final template with a much better SNR than the linear stack. \\

Bostock \textit{et al.} (2015 ~\cite{BOS_2015}) studied the magnitudes of low-frequency earthquakes (LFE) templates below southern Vancouver Island. They computed the magnitudes from the waveforms using the ray approximation, and observed that the magnitude-frequency distribution was better represented by a power law, with a b-value ($\sim$ 6.3) much higher than what is observed for regular earthquakes. They assumed that the source pulse duration is measured by the reciprocal of the instantaneous frequency, and observed a weak scaling between seismic moment and duration. They observed that the ratio of slip between two template waveforms is much higher than the ratio of pulse duration (7.36 and 1.29) , and concluded that there is no self-similarity for LFE and that larger moment events appear to be the result of increased slip. To reconcile the scaling between magnitude and frequency, and the scaling between seismic moment and slip, they proposed that multiple independently slipping sources are present within the same LFE template.  The scaling of LFE would thus be different from both large scale slow slip events (SSE) and regular earthquakes. \\

Houston (2015 ~\cite{HOU_2015}) studied the sensitivity of tremor to tidal stress. She divided tremor into two groups: tremors arriving before 1.5 days after the tremor front, and tremors arriving after 1.5 days after the tremor front. She computed the evolution of tidal stress within the tremor region, and computed for each point in the regional grid the ratio of tremors occurring at a given level of tidal stress divided by the total number of tremors recorded at this grid point. She noticed a much stronger correlation between tremor activity and tidal stress changes after the passage of the tremor front. She interpreted this phenomenon with a stress threshold failure model. There is a big stress increase on the fault with the arrival of the tremor front, such that the stress stays much higher than the fault strength even when the tidal stress varies. It generates a lot of tremor, but a weak influence of tides on tremor activity. After the passage of the tremor front, tides cause small variations of stress on a weaker fault, such that there is an alternance of states with fault strength higher than stress on fault, and fault strength lower than stress on fault with each tidal cycle. Thus, there are less tremors, but a stronger influence of tides on tremor activity. \\

Hyndman \textit{et al.} (2015 ~\cite{HYN_2015}) investigated the processes that control the position of Episodic Tremor and Slip (ETS) in the Cascadia subduction zone. They noticed that the high temperatures in the young subducting oceanic plate, the geodetic data, and the recordings of coseismic subsidence in buried coastal marshes during past great earthquakes, all point out to a downdip limit of the seismogenic zone located offshore. The position of the slow slip and the tremor is well known, although the depths have some uncertainty. The slip may extend seaward of the tremor, but there is a clear separation between the seismogenic zone and the ETS zone, with the ETS zone being located about 70 km east of the downdip of the seismogenic zone, and the volcanic arc being located about 100 km east of the ETS zone. A previous study showed that the position of the subduction zone ETS does not coincide with a specific temperature or dehydration reaction. The authors pointed out that ETS has been related to high pore fluid pressures close to the plate boundary. They argued that the bending of the subducting plate at the ocean trench may introduce a large amount of water in the upper oceanic mantle, resulting in extensive serpentinization. Moreover, the serpentinization of the fore-arc mantle corner may increase its vertical impermeability, while keeping a high permeability parallel to the fault, thus channelling all the fluid updip in the subducting oceanic crust. The dehydration of the serpentinite from the upper oceanic mantle, and the focusing of rising fluids along the plate boundary should result in large amounts of fluids available at the fore-arc mantle corner. Additionally, there seems to be a good coincidence between the location of the fore-arc mantle corner, and the location of ETS. The authors then observed that the deep fore-arc crust has a very low Poisson's ratio (less than 0.22), and that the only mineral with a very low Poisson's ratio is quartz (about 0.1), which led them to conclude that there may be a significant amount of quartz (about 10 \% in volume) in the deep fore-arc crust above the fore-arc mantle. Moreover, as the solubility of silica increases with temperature, fluids generated at depth and rising up the subduction channel should be rich in silica. The authors concluded that there may be a relation between quartz veins formation in the deep fore-arc crust and ETS. However, several constraints as the magnitude and mechanism of the low-frequency earthquakes, and the vertical extent of the tremor should be explained. \\

Plourde \textit{et al.} (2015 ~\cite{PLO_2015}) have detected low-frequency earthquakes (LFEs) in Northern California during the April 2008 Episodic Tremor and Slip (ETS) event using seismic data from the EarthScope Flexible Array Mendocino Experiment (FAME). They used a combination of autodetection methods and visual identification to obtain the initial templates. Then, they recovered higher signal-to-noise (SNR) LFE signals using iterative network cross correlation. They found that the LFE families were located above the plate boundary, with a large distribution of depths (28-47 km). Three additional families were found on the Maacama and Bucknell Creek faults. On these faults, LFEs tend to occur in bursts, while repeating earthquakes occur as single events or in small groups. LFEs and earthquakes have also different frequency contents. They conclude that dehydration of the mantle and further upward migration of water through the deep crustal fault system could explain the generation of both tremor and regular seismicity on these two faults. \\

Frank \textit{et al.} (2016 ~\cite{FRA_2016}) carried out a statistical analysis of a catalog of low-frequency earthquakes (LFEs) recorded between January 2005 and April 2007 in the Guerrero, Mexico, subduction zone. There are two sources of LFEs: in the ``transient zone'', most of the LFEs occur in bursts during slow slip events, while the LFE activity is much lower during inter-slow slip periods ; in the ``sweet spot'', bursts of LFEs are emitted nearly continuously. The authors translated the catalog for each LFE family into a discrete time series by binning each cataloged event into the one-minute-long time step in which it is observed. They then computed the autocorrelation sequence and the spectral density function of the time series for two 4-month-long windows, one corresponding to an inter-slow slip period, and one corresponding to the 2006 slow slip event. They observed that in the transient zone, LFEs behave as an homogeneous Poisson's process during the inter-slow slip period, and as a long memory process during the co-slow slip period. They then computed the slope of the logarithm of the spectral density function for ten-day-long sliding windows over the whole catalog. For the transient zone, they observed an increase in the slope for each geodetically detected slow slip event. For the sweet spot, the slope stay higher than zero between geodetically detected slow slip events, implying that there may be smaller slow slip events that have yet to be observed. Moreover, the authors computed the cross correlation between the event count time series for each LFE family. They noted that there is a strong correlation in the sweet spot all the time, but that in the transition zone, the interaction is weak during the inter-slow slip period, and strong during the co-slow slip period. Finally, the authors designed a simple interaction model where the event rate at each asperity is modeled by an homogeneous Poisson's process. The time between two events is then modified by two phenomena: first, a migrating pulse decreases the inter-event time when it reaches the asperity ; second, the inter-event time decreases when an event occurs at a nearby asperity. They observed that a migrating pulse alone cannot reproduce the slope of the spectral density function, but that a high enough asperity density can reproduce the slope. They concluded that asperities at the plate interface may be locked during the inter-slow slip period, and that a new mechanism such as migrating pore pressure pulses may occur during slow slip, and activate the asperities. \\

Gomberg \textit{et al.} (2016 ~\cite{GOM_2016_GRL}) studied the relationship between seismic moment and duration for fast and slow earthquakes population. They used GPS data and tremor catalogs in Japan and Cascadia for slow slip events, and crustal earthquakes from the SRCMOD database for fast slip events. They distinguished between unbounded events, for which fault growth is two-dimensional and moment is proportional to the cube of duration, and bounded events, for which fault growth is one dimensional and moment is proportional to duration. The proposed dislocation model does not require different scaling between fast and slow earthquakes. Instead, there is a continuous but bimodal distribution of slip modes: elastic, velocity-weakening patches generate fast slip, while viscous, velocity-strengthening background generates slow, aseismic slip. The size and distribution of patches on a fault determinate the dominant mode. \\

Chestler and Creager (2017 ~\cite{CHE_2017_JGR}) gathered a catalog of low-frequency earthquakes (LFEs) in the Olympic Peninsula, Washington. They used data from the Array of Arrays (AofA) and the Cascadia Arrays for Earthscope (CAFE) experiments, and stacked the seismograms over stations and arrays to obtain a signal with high signal-to-noise ratio (SNR). They used the peaks in the signal as potential times for LFE occurrence, and stacked the autocorrelation signals to verify the repetition in time of potential LFEs. They then inverted for the seismic moment of the LFEs using the time integral of the displacement pulse for the north and east channels, and by computing the moment rate from the displacement amplitude history. They thus gathered a catalog of 34,264 LFEs from 43 families. By analyzing the moment-frequency distribution of the LFEs, they observed that an exponential law better fitted the data than a power law, and concluded that the seismic moment of LFEs is scale-limited, either by the amount of slip, either by the area of slip. They proposed two end member rupture models: in the first one, the same unique patch slips all the time; in the second one, each LFE corresponds to a different subpatch that slips only once. However, they noted that to obtain a stress drop in the same range as the one obtained for slow slip models from geodetic inversions, the number of subpatches must remain low, and is unlikely to be higher than 10. \\

Chestler and Creager (2017 ~\cite{CHE_2017_G3}). \\

Shelly (2017 ~\cite{SHE_2017}) assembled a catalog of more than one million low-frequency earthquakes (LFEs) recorded along the central San Andreas fault between 2001 and 2015. The waveform templates for the 88 LFE families were developed by Shelly and Hardebeck (2010 ~\cite{SHE_2010}) using cross correlations of seismograms from the High-Resolution Seismic Network (HRSM) borehole network installed in the vicinity of Parkfield, California. The best 100 LFEs were linearly stacked to form a high signal-to-noise ratio for each family. Event detection was then carried out over 15 years using a multichannel matched filter method. Two thresholds were used: the mean cross correlation coefficient across all channels must be higher than 0.16, and the sum of all cross correlation coefficients across all channels must be higher than 4.0. An increase in the LFE event rate was observed after the 2004 Parkfield earthquake. A large diversity of recurrence behaviours was observed among the LFE families, from semicontinuous to highly episodic. Particularly, two families exhibited bimodal recurrence patterns (about 3 and 6 days for the first one, and about 2 and 4 days for the second one). Fast (15 to 90 km/h) and slow (5-15 km/h) migrations of the LFEs were observed along the San Andreas fault. False detections may occur and can be eliminated by using a higher detection threshold for the cross correlation. The detection rate may vary along time due to station outage. Finally, the average cross correlation of detected events could be used for network monitoring. \\

\end{document}
