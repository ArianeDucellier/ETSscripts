\documentclass[main.tex]{subfiles}
 
\begin{document}

\part{Time lags}

\chapter{Data}

\section{Description of the dataset}

The available data come from eight small-aperture arrays installed in the eastern part of the Olympic Peninsula. The aperture of the arrays is about 1 km, and station spacing is a few hundred meters. The arrays are around 5 to 10 km apart from each other. Most of the arrays were installed for nearly a year and were able to record the main August 2010 ETS event, and some of the stations were able to record the August 2011 ETS event. The time and locations of the tremor sources were all computed by Ghosh \textit{et al.} (2012 ~\cite{GHO_2012}).

\section{Websites to access the data}

\subsection{FDSN}

\href{http://www.fdsn.org/networks/}{\textbf{International Federation of Digital Seismograph Networks}} \\

This website gives a list of the network codes, and the corresponding map, with the names and locations of stations. The selected experiments are:
\begin{itemize}
	\item XG (2009-2011): Cascadia Array of Arrays
	\item XU (2006-2012): Collaborative Research: Earthscope integrated investigations of Cascadia subduction zone tremor, structure and process
\end{itemize}

The stations are the following:
\begin{itemize}
	\item Port Angeles XG - PA01 to PA13
	\item Danz Ranch XG - DR01 to DR10, and DR12
	\item Lost Cause XG - LC01 to LC14
	\item Three Bumps XG - TB01 toTB14
	\item Burnt Hill XG - BH01 to BH11
	\item Cat Lake XG - CL01 to CL20
	\item Gold Creek XG - GC01 to GC14
	\item Blyn XU - BS01 to BS06, BS11, BS20 to BS27
\end{itemize}

\subsection{IRIS}

\href{http://ds.iris.edu/mda}{\textbf{IRIS DMC MetaData Aggregator}} \\

This website gives for each station:
\begin{itemize}
	\item Location and time of recording
	\item Epoch (effective periods of recording during the time that the station was installed)
	\item Type of instrument
	\item Channels
\end{itemize}

The data can be downloaded from the IRIS DMC using the Python package obspy, and the subpackage obspy.clients.fdsn, or alternatively, they can be downloaded from the ESS server Rainier, using the subpackage obspy.clients.earthworm.

\subsection{PNSN}

\href{https://www.pnsn.org/tremor}{\textbf{Pacific Northwest Seismic Network tremor catalog}} \\

This website gives the dates and locations of tremor activity in Cascadia. Following Ghosh \textit{et al.} (2012 ~\cite{GHO_2012}), the selected periods of tremors are:
\begin{itemize}
	\item From November 9th 2009 to November 13th 2009,
	\item From March 16th 2010 to March 21st 2010,
	\item From August 14th 2010 to August 22nd 2010.
\end{itemize}

\chapter{Method}

\chapter{Results}

\chapter{Discussion and things to do}

For each tremor window, compare the peak CC with the RMS outside the selected time window (8 to 10 s). \\

Try some sort of clustering of CC windows.

\end{document}
