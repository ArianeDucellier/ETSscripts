\documentclass[main.tex]{subfiles}
 
\begin{document}

\part{Time lags}

\chapter{Method}

Some bla bla about cross correlaion and stacking

\chapter{Cross-correlation of tremor recordings}

\section{Definitions}

We define the Fourier transform $\hat{f}$ of the function $f$ by:

\begin{equation}
\hat{f} (\omega) = \frac{1}{\sqrt{2 \pi}} \int_{- \infty}^{\infty} f(t) e^{-i \omega t} dt
\end{equation}

\begin{equation}
f(t) = \frac{1}{\sqrt{2 \pi}} \int_{- \infty}^{\infty} \hat{f} (\omega) e^{i \omega t} d\omega
\end{equation}

We define the convolution product by:

\begin{equation}
(f * g) (t) = \int_{-\infty}^{\infty} f(\tau) g(t - \tau) d\tau
\end{equation}

We have:

\begin{equation}
\hat{(f * g)} (\omega) = \sqrt{2 \pi} \hat{f} (\omega) \hat{g} (\omega)
\end{equation}

and:

\begin{equation}
\hat{(f g)} (\omega) = \frac{1}{\sqrt{2 \pi}} \hat{f} (\omega) * \hat{g} (\omega)
\end{equation}

We define the cross correlation by:

\begin{equation}
(f \otimes g) (t) = \int_{- \infty}^{\infty} f^* (\tau) g(t + \tau) d\tau
\end{equation}

where $f^*$ is the complex conjugate of $f$. We have:

\begin{equation}
\hat{(f \otimes g)} (\omega) = \sqrt{2 \pi} \hat{f}^* (\omega) \hat{g} (\omega)
\end{equation}
 
 and:

\begin{equation}
\hat{(f g)} (\omega) = \frac{1}{\sqrt{2 \pi}} \hat{f}^* (\omega) \otimes \hat{g} (\omega)
\end{equation}

Finally, we have:

\begin{equation}
(f(t) * g(t)) \otimes (f(t) * g(t)) = (f(t) \otimes f(t)) * (g(t) \otimes g(t))
\end{equation}

If we have a point source at $\xi$ with a source function $f_k (t)$ for $k = 1, 2, 3$, then we can express the displacement $u(x, t)$ with the Green's function:

\begin{equation}
u_i(x, t) = \int_{- \infty}^{\infty} G_{ik} (x, \xi, t - \tau) f_k (\xi, \tau) d\tau
\end{equation}

In the case of a moment tensor, we have:

\begin{equation}
u_i(x, t) = \int_{- \infty}^{\infty} \frac{\partial G_{ip} }{\partial \xi_q} (x, \xi, t - \tau) M_{pq} (\xi, \tau) d\tau
\end{equation}

In the Fourier domain, we have:

\begin{equation}
\hat{u_i} (x, \omega) = \sqrt{2 \pi} \hat{G_{ik}} (x, \xi, \omega) \hat{f_k} (\xi, \omega)
\end{equation}

or:

\begin{equation}
\hat{u_i} (x, \omega) = \sqrt{2 \pi} \hat{\frac{\partial G_{ip}}{\partial \xi_q} } (x, \xi, \omega) \hat{M_{pq}} (\xi, \omega)
\end{equation}

When we compute the cross correlation between two components of the displacements, we have:

\begin{equation}
(u_i \otimes u_j) (x, t) = \int_{- \infty}^{\infty} u_i^* (x, \tau) u_j (x, t + \tau) d\tau
\end{equation}

In the Fourier domain, we have:

\begin{equation}
\begin{split}
\hat{(u_i \otimes u_j)} (x, \omega) & = \sqrt{2 \pi} \hat{u_i}^* (x, \omega) \hat{u_j} (x, \omega) \\
                                        & = (2 \pi)^{\frac{3}{2}} [\hat{G_{ik}}^* (x, \xi, \omega) \hat{f_k}^* (\xi, \omega)] [\hat{G_{jl}} (x, \xi, \omega) \hat{f_l} (\xi, \omega)] \\
                                        & = (2 \pi)^{\frac{3}{2}} [\hat{\frac{\partial G_{ip}}{\partial \xi_q}}^* (x, \xi, \omega) \hat{M_{pq}}^* (\xi, \omega)] [\hat{\frac{\partial G_{jr}}{\partial \xi_s}} (x, \xi, \omega) \hat{M_{rs}} (\xi, \omega)] 
\end{split}
\end{equation}

\section{Change of coordinates}

\subsection{Point source}

Equation (15) can be written as:

\begin{equation}
\begin{pmatrix}
\hat{(u_1 \otimes u_1)} & \hat{(u_1 \otimes u_2)} & \hat{(u_1 \otimes u_3)} \\
\hat{(u_2 \otimes u_1)} & \hat{(u_2 \otimes u_2)} & \hat{(u_2 \otimes u_3)} \\
\hat{(u_3 \otimes u_1)} & \hat{(u_3 \otimes u_2)} & \hat{(u_3 \otimes u_3)}
\end{pmatrix} = (2 \pi)^{\frac{3}{2}}
\begin{pmatrix}
\hat{G_{11}}^* & \hat{G_{12}}^* & \hat{G_{13}}^* \\
\hat{G_{21}}^* & \hat{G_{22}}^* & \hat{G_{23}}^* \\
\hat{G_{31}}^* & \hat{G_{32}}^* & \hat{G_{33}}^*
\end{pmatrix}
\begin{pmatrix}
\hat{f_1}^* \\
\hat{f_2}^* \\
\hat{f_3}^*
\end{pmatrix}
\begin{pmatrix}
\hat{f_1} & \hat{f_2} & \hat{f_3}
\end{pmatrix}
\begin{pmatrix}
\hat{G_{11}} & \hat{G_{21}} & \hat{G_{31}} \\
\hat{G_{12}} & \hat{G_{22}} & \hat{G_{32}} \\
\hat{G_{13}} & \hat{G_{23}} & \hat{G_{33}}
\end{pmatrix}
\end{equation}

that is:

\begin{equation}
\hat{U} = (2 \pi)^{\frac{3}{2}} \hat{G}^* \hat{f}^* \hat{f}^T \hat{G}^T
\end{equation}

We define a new coordinate system with the unit vectors $n^{(1)}$, $n^{(2)}$ and $n^{(3)}$, and the matrix $N$ by:

\begin{equation}
N = \begin{pmatrix}
n_1^{(1)} & n_1^{(2)} & n_1^{(3)} \\
n_2^{(1)} & n_2^{(2)} & n_2^{(3)} \\
n_3^{(1)} & n_3^{(2)} & n_3^{(3)}
\end{pmatrix}
\end{equation}

In the new coordinate system, the Green's function is equal to $G' = N^T G N$, thus we have:

\begin{equation}
N^T \hat{U} N = (2 \pi)^{\frac{3}{2}} \hat{G'}^* N^T \hat{f}^* \hat{f}^T N \hat{G'}^T
\end{equation}

with:

\begin{equation}
N^T \hat{U} N = \begin{pmatrix}
\hat{(u . n^{(1)} \otimes u . n^{(1)})} & \hat{(u . n^{(1)} \otimes u . n^{(2)})} & \hat{(u . n^{(1)} \otimes u . n^{(3)})} \\
\hat{(u . n^{(2)} \otimes u . n^{(1)})} & \hat{(u . n^{(2)} \otimes u . n^{(2)})} & \hat{(u . n^{(2)} \otimes u . n^{(3)})} \\
\hat{(u . n^{(3)} \otimes u . n^{(1)})} & \hat{(u . n^{(3)} \otimes u . n^{(2)})} & \hat{(u . n^{(3)} \otimes u . n^{(3)})}
\end{pmatrix}
\end{equation}

If we choose $N_1$ such that:

\begin{equation}
N_1^T \hat{f} = \begin{pmatrix}
\hat{F} \\
0 \\
0
\end{pmatrix}
\end{equation}

we get:

\begin{equation}
N_1^T \hat{U} N_1 = (2 \pi)^{\frac{3}{2}} \hat{F}^* \hat{F} \begin{pmatrix}
\hat{G'_{11}}^* \hat{G'_{11}} & \hat{G'_{11}}^* \hat{G'_{21}} & \hat{G'_{11}}^* \hat{G'_{31}} \\
\hat{G'_{21}}^* \hat{G'_{11}} & \hat{G'_{21}}^* \hat{G'_{21}} & \hat{G'_{21}}^* \hat{G'_{31}} \\
\hat{G'_{31}}^* \hat{G'_{11}} & \hat{G'_{31}}^* \hat{G'_{21}} & \hat{G'_{31}}^* \hat{G'_{31}}
\end{pmatrix}
\end{equation}

If we choose $N_2$ such that:

\begin{equation}
N_2^T \hat{f} = \begin{pmatrix}
0 \\
\hat{F} \\
0
\end{pmatrix}
\end{equation}

we get:

\begin{equation}
N_2^T \hat{U} N_2 = (2 \pi)^{\frac{3}{2}} \hat{F}^* \hat{F} \begin{pmatrix}
\hat{G'_{12}}^* \hat{G'_{12}} & \hat{G'_{12}}^* \hat{G'_{22}} & \hat{G'_{12}}^* \hat{G'_{32}} \\
\hat{G'_{22}}^* \hat{G'_{12}} & \hat{G'_{22}}^* \hat{G'_{22}} & \hat{G'_{22}}^* \hat{G'_{32}} \\
\hat{G'_{32}}^* \hat{G'_{12}} & \hat{G'_{32}}^* \hat{G'_{22}} & \hat{G'_{32}}^* \hat{G'_{32}}
\end{pmatrix}
\end{equation}

If we choose $N_3$ such that:

\begin{equation}
N_3^T \hat{f} = \begin{pmatrix}
0 \\
0 \\
\hat{F}
\end{pmatrix}
\end{equation}

we get:

\begin{equation}
N_3^T \hat{U} N_3 = (2 \pi)^{\frac{3}{2}} \hat{F}^* \hat{F} \begin{pmatrix}
\hat{G'_{13}}^* \hat{G'_{13}} & \hat{G'_{13}}^* \hat{G'_{23}} & \hat{G'_{13}}^* \hat{G'_{33}} \\
\hat{G'_{23}}^* \hat{G'_{13}} & \hat{G'_{23}}^* \hat{G'_{23}} & \hat{G'_{23}}^* \hat{G'_{33}} \\
\hat{G'_{33}}^* \hat{G'_{13}} & \hat{G'_{33}}^* \hat{G'_{23}} & \hat{G'_{33}}^* \hat{G'_{33}}
\end{pmatrix}
\end{equation}

If we define strike $\phi$, dip $\delta$ and rake $\lambda$, we can define the following vectors:

\begin{equation}
e_1 = \begin{pmatrix}
\sin \phi \\
\cos \phi \\
0
\end{pmatrix} \text{, } e_2 = \begin{pmatrix}
\cos \phi \\
- \sin \phi \\
0
\end{pmatrix} \text{, } e_3 = \cos \delta e_2 - \sin \delta e_z = \begin{pmatrix}
\cos \phi \cos \delta \\
- \sin \phi \cos \delta \\
- \sin \delta
\end{pmatrix} \text{ and } e_4 = \sin \delta e_2 + \cos \delta e_z = \begin{pmatrix}
\cos \phi \sin \delta \\
- \sin \phi \sin \delta \\
\cos \delta
\end{pmatrix}
\end{equation}

and the new coordinate system $(u, v, w)$ with:

\begin{equation}
u = \cos \lambda e_1 - \sin \lambda e_3 \text{, } v = - \sin \lambda e_1 - \cos \lambda e_3 \text{ and } w = e_4
\end{equation}

Thus we have:

\begin{equation}
u = \begin{pmatrix}
\sin \phi \cos \lambda - \cos \phi \cos \delta \sin \lambda \\
\cos \phi \cos \lambda + \sin \phi \cos \delta \sin \lambda \\
\sin \delta \sin \lambda
\end{pmatrix} \text{, } v = \begin{pmatrix}
- \sin \phi \sin \lambda - \cos \phi \cos \delta \cos \lambda \\
- \cos \phi \sin \lambda + \sin \phi \cos \delta \sin \lambda \\
- \sin \delta \cos \lambda
\end{pmatrix} \text{ and } w = \begin{pmatrix}
\cos \phi \sin \delta \\
- \sin \phi \sin \delta \\
\cos \delta
\end{pmatrix}
\end{equation}

We can choose:

\begin{equation}
N_1 = \begin{pmatrix}
u_x & v_x & w_x \\
u_y & v_y & w_y \\
u_z & v_z & w_z
\end{pmatrix} = \begin{pmatrix}
\sin \phi \cos \lambda - \cos \phi \cos \delta \sin \lambda & - \sin \phi \sin \lambda - \cos \phi \cos \delta \cos \lambda & \cos \phi \sin \delta \\
\cos \phi \cos \lambda + \sin \phi \cos \delta \sin \lambda & - \cos \phi \sin \lambda + \sin \phi \cos \delta \sin \lambda & - \sin \phi \sin \delta \\
\sin \delta \sin \lambda & - \sin \delta \cos \lambda & \cos \delta
\end{pmatrix}
\end{equation}

to get Equation (19). To get Equations (21) and (23), we just have to permute the columns of $N_1$ to get $N_2$ and $N_3$.

If we go back into the time domain, we have:

\begin{equation}
N_k^T \begin{pmatrix}
u_1 \otimes u_1 & u_1 \otimes u_2 & u_1 \otimes u_3 \\
u_2 \otimes u_1 & u_2 \otimes u_2 & u_2 \otimes u_3 \\
u_3 \otimes u_1 & u_3 \otimes u_2 & u_3 \otimes u_3
\end{pmatrix} N_k = \begin{pmatrix}
(F * G'_{1k}) \otimes (F * G'_{1k}) & (F * G'_{1k}) \otimes (F * G'_{2k}) & (F * G'_{1k}) \otimes (F * G'_{3k}) \\
(F * G'_{2k}) \otimes (F * G'_{1k}) & (F * G'_{2k}) \otimes (F * G'_{2k}) & (F * G'_{2k}) \otimes (F * G'_{3k}) \\
(F * G'_{3k}) \otimes (F * G'_{1k}) & (F * G'_{3k}) \otimes (F * G'_{2k}) & (F * G'_{3k}) \otimes (F * G'_{3k})
\end{pmatrix}
\end{equation}

which can also be written as:

\begin{equation}
N_k^T \begin{pmatrix}
u_1 \otimes u_1 & u_1 \otimes u_2 & u_1 \otimes u_3 \\
u_2 \otimes u_1 & u_2 \otimes u_2 & u_2 \otimes u_3 \\
u_3 \otimes u_1 & u_3 \otimes u_2 & u_3 \otimes u_3
\end{pmatrix} N_k = \begin{pmatrix}
(F \otimes F) * (G'_{1k} \otimes G'_{1k}) & (F \otimes F) * (G'_{1k} \otimes G'_{2k}) & (F^S \otimes F^S) * (G'_{1k} \otimes G'_{3k}) \\
(F \otimes F) * (G'_{2k} \otimes G'_{1k}) & (F \otimes F) * (G'_{2k} \otimes G'_{2k}) & (F^S \otimes F^S) * (G'_{2k} \otimes G'_{3k}) \\
(F \otimes F) * (G'_{3k} \otimes G'_{1k}) & (F \otimes F) * (G'_{3k} \otimes G'_{2k}) & (F^S \otimes F^S) * (G'_{3k} \otimes G'_{3k})
\end{pmatrix}
\end{equation}

\subsection{Moment tensor}

We have:

\begin{equation}
M_{pq} = \int \int_{\Sigma} m_{pq} d\Sigma = \int \int_{\Sigma} \mu (\nu_p [u_q] + \nu_q [u_p]) d\Sigma = \mu A (\nu_p [u_q] + \nu_q [u_p])
\end{equation}

where $\nu$ is the normal to the fault surface and$[u]$ is the displacement discontinuity on the fault.

We define a new coordinates system with the unit vectors $n^{(1)}$, $n^{(2)}$ and $n^{(3)}$, and the matrix $N$ by:

\begin{equation}
N = \begin{pmatrix}
n_1^{(1)} & n_1^{(2)} & n_1^{(3)} \\
n_2^{(1)} & n_2^{(2)} & n_2^{(3)} \\
n_3^{(1)} & n_3^{(2)} & n_3^{(3)}
\end{pmatrix}
\end{equation}

In the new coordinates system, the Green's function is equal to $G' = N^T G N$, the moment tensor to $M' = N^T M N$, and the displacement to $u' = N^T u$. We choose N such that:

\begin{equation}
M' = \mu D A \begin{pmatrix}
1 & 0 & 0 \\
0 & -1 & 0 \\
0 & 0 & 0
\end{pmatrix}
\end{equation}

Therefore, we have:

\begin{equation}
\hat{u'_i \otimes u'_j} = (2 \pi)^{\frac{3}{2}} \mu D^* A [\hat{\frac{\partial G'_{i1}}{\partial \xi'_1}}^* - \hat{\frac{\partial G'_{i2}}{\partial \xi'_2}}^*] \mu D A [\hat{\frac{\partial G'_{j1}}{\partial \xi'_1}} - \hat{\frac{\partial G'_{j2}}{\partial \xi'_1}}]
\end{equation}

If we come back in the time domain, we have:

\begin{equation}
u'_i \otimes u'_j = \mu^2 A^2 (D * [\frac{\partial G'_{i1}}{\partial \xi'_1} - \frac{\partial G'_{i2}}{\partial \xi'_2}]) \otimes (D * [\frac{\partial G'_{j1}}{\partial \xi'_1} - \frac{\partial G'_{j2}}{\partial \xi'_2}])
\end{equation}

which can also be written as:

\begin{equation}
u'_i \otimes u'_j = \mu^2 A^2 (D \otimes D) * ([\frac{\partial G'_{i1}}{\partial \xi'_1} - \frac{\partial G'_{i2}}{\partial \xi'_2}] \otimes [\frac{\partial G'_{j1}}{\partial \xi'_1} - \frac{\partial G'_{j2}}{\partial \xi'_2}])
\end{equation}

If we choose N such that:

\begin{equation}
M' = \mu D A \begin{pmatrix}
1 & 0 & 0 \\
0 & 0 & 0 \\
0 & 0 & -1
\end{pmatrix}
\end{equation}

we would get:

\begin{equation}
u'_i \otimes u'_j = \mu^2 A^2 (D \otimes D) * ([\frac{\partial G'_{i1}}{\partial \xi'_1} - \frac{\partial G'_{i3}}{\partial \xi'_3}] \otimes [\frac{\partial G'_{j1}}{\partial \xi'_1} - \frac{\partial G'_{j3}}{\partial \xi'_3}])
\end{equation}

If we choose N such that:

\begin{equation}
M' = \mu D A \begin{pmatrix}
0 & 0 & 0 \\
0 & 1 & 0 \\
0 & 0 & -1
\end{pmatrix}
\end{equation}

we would get:

\begin{equation}
u'_i \otimes u'_j = \mu^2 A^2 (D \otimes D) * ([\frac{\partial G'_{i2}}{\partial \xi'_2} - \frac{\partial G'_{i3}}{\partial \xi'_3}] \otimes [\frac{\partial G'_{j2}}{\partial \xi'_2} - \frac{\partial G'_{j3}}{\partial \xi'_3}])
\end{equation}

\subsubsection{How to compute the autocorrelation of the source term $F \otimes F$ or $D \otimes D$?}

\paragraph{White noise}

At time $t_i$, we assume that the amplitude of the source function is $A_i$ (random variable with expectancy $m = 0$ and standard deviation $\sigma$). We define $a_i$ by:

\begin{equation}
P (A_i < p_0) = \int_{- \infty}^{p_0} a_i (p) dp
\end{equation}

We have:

\begin{equation}
\int_{- \infty}^{\infty} a_i (p) dp = m
\end{equation}

and:

\begin{equation}
\int_{- \infty}^{\infty} a_i^2 (p) dp = m^2 + \sigma ^2
\end{equation}

We suppose that the source function is a white noise, that is:

\begin{equation}
\int_{- \infty}^{\infty} (a_i (p) - m) (a_j (p) - m) dp = 0
\end{equation}

Thus, we have:

\begin{equation}
\int_{- \infty}^{\infty} a_i (p) a_j (p) dp = m^2
\end{equation}

We define the source time function $F (t_i) = A_i$ and we compute the autocorrelation. We have:

\begin{equation}
(F \otimes F) (t) = \int_{- \infty}^{\infty} F^* (\tau) F (t + \tau) d\tau
\end{equation}

The expectancy of the term $F^* (\tau) F (t + \tau)$ is $m^2 + \sigma^2$ if $t = 0$ and $m^2$ if $t \neq 0$.

\section{Stacking}

\chapter{Computation of amplitudes for P-, SV- and SH-waves}

\section{Changes of coordinates}

We denote $(x, y)$ the coordinates of the receiver array, $(x_0, y_0)$ the coordinates of the tremor source, and $d$ the depth of the tremor source.

In the $(\vec{e}_X, \vec{e}_Y, \vec{e}_Z)$ coordinate system, we have:

\begin{equation}
\vec{e}_R = \begin{pmatrix}
\cos \beta \\
\sin \beta \\
0
\end{pmatrix} \text{, } \vec{e}_T = \begin{pmatrix}
- \sin \beta \\
\cos \beta \\
0
\end{pmatrix} \text{ and } \vec{e}_Z = \begin{pmatrix}
0 \\
0 \\
1
\end{pmatrix}
\end{equation}

with $\beta = \atantwo (y - y_0 , x - x_0)$

In the $(\vec{e}_R, \vec{e}_T, \vec{e}_Z)$ coordinate system, we have:

\begin{equation}
\vec{e}_X = \begin{pmatrix}
\cos \beta \\
- \sin \beta \\
0
\end{pmatrix} \text{, } \vec{e}_Y = \begin{pmatrix}
\sin \beta \\
\cos \beta \\
0
\end{pmatrix} \text{ and } \vec{e}_Z = \begin{pmatrix}
0 \\
0 \\
1
\end{pmatrix}
\end{equation}

For the direct wave, we have:

\begin{equation}
\vec{e}_P = \begin{pmatrix}
\sin \alpha \\
0 \\
\cos \alpha
\end{pmatrix} \text{, } \vec{e}_{SV} = \begin{pmatrix}
\cos \alpha \\
0 \\
- \sin \alpha
\end{pmatrix} \text{ and } \vec{e}_{SH} = \begin{pmatrix}
0 \\
1 \\
0
\end{pmatrix}
\end{equation}

with $\alpha = \frac{\sqrt{ (x - x_0)^2 + (y - y_0)^2}}{d}$

For the reflected wave, we have:

\begin{equation}
\vec{e}_P = \begin{pmatrix}
\sin \alpha \\
0 \\
- \cos \alpha
\end{pmatrix} \text{, } \vec{e}_{SV} = \begin{pmatrix}
\cos \alpha \\
0 \\
\sin \alpha
\end{pmatrix} \text{ and } \vec{e}_{SH} = \begin{pmatrix}
0 \\
- 1 \\
0
\end{pmatrix}
\end{equation}

where $\alpha$ is computed with the RayTracing code.

In the $(\vec{e}_P, \vec{e}_{SV}, \vec{e}_{SH})$ coordinate system, we have for the direct wave:

\begin{equation}
\vec{e}_R = \begin{pmatrix}
\sin \alpha \\
\cos \alpha \\
0
\end{pmatrix} \text{, } \vec{e}_T = \begin{pmatrix}
0 \\
0 \\
1
\end{pmatrix} \text{ and } \vec{e}_Z = \begin{pmatrix}
\cos \alpha \\
- \sin \alpha \\
0
\end{pmatrix}
\end{equation}

For the reflected wave, we have:

\begin{equation}
\vec{e}_R = \begin{pmatrix}
\sin \alpha \\
\cos \alpha \\
0
\end{pmatrix} \text{, } \vec{e}_T = \begin{pmatrix}
0 \\
0 \\
- 1
\end{pmatrix} \text{ and } \vec{e}_Z = \begin{pmatrix}
- \cos \alpha \\
\sin \alpha \\
0
\end{pmatrix}
\end{equation}

We compute the seismic moment $M$ in the $(\vec{e}_X, \vec{e}_Y, \vec{e}_Z)$ coordinate system. We have:

\begin{equation}
M_{ij} = u_i \nu_j + u_j \nu_i
\end{equation}

with:

\begin{equation}
\vec{u} = \begin{pmatrix}
- \cos \delta \cos \phi \\
\cos \delta \sin \phi \\
\sin \delta
\end{pmatrix} \text{ and } \vec{\nu} = \begin{pmatrix}
\sin \delta \cos \phi \\
- \sin \delta \sin \phi \\
\cos \delta 
\end{pmatrix}
\end{equation}

where $\phi$ is the strike of the subducting plate, and $\delta$ is the dip of the subducting plate.

We then compute the value of $M$ in the $(\vec{e}_P, \vec{e}_{SV}, \vec{e}_{SH})$ coordinate system. We have:

\begin{equation}
\begin{pmatrix}
M_{RR} & M_{RT} & M_{RZ} \\
M_{TR} & M_{TT} & M_{TZ} \\
M_{ZR} & M_{ZT} & M_{ZZ}
\end{pmatrix} = \begin{pmatrix}
\cos \beta & \sin \beta & 0 \\
- \sin \beta & \cos \beta & 0 \\
0 & 0 & 1
\end{pmatrix} \begin{pmatrix}
M_{XX} & M_{XY} & M_{XZ} \\
M_{YX} & M_{YY} & M_{YZ} \\
M_{ZX} & M_{ZY} & M_{ZZ}
\end{pmatrix} \begin{pmatrix}
\cos \beta & - \sin \beta & 0 \\
\sin \beta & \cos \beta & 0 \\
0 & 0 & 1
\end{pmatrix}
\end{equation}

For the direct wave, we have:

\begin{equation}
\begin{pmatrix}
M_{PP} & M_{PSV} & M_{PSH} \\
M_{SVP} & M_{SVSV} & M_{SVSH} \\
M_{SHP} & M_{SHSV} & M_{SHSH}
\end{pmatrix} = \begin{pmatrix}
\sin \alpha & 0 & \cos \alpha \\
\cos \alpha & 0 & - \sin \alpha \\
0 & 1 & 0
\end{pmatrix} \begin{pmatrix}
M_{RR} & M_{RT} & M_{RZ} \\
M_{TR} & M_{TT} & M_{TZ} \\
M_{ZR} & M_{ZT} & M_{ZZ}
\end{pmatrix} \begin{pmatrix}
\sin \alpha & \cos \alpha & 0 \\
0 & 0 & 1 \\
\cos \alpha & - \sin \alpha & 0
\end{pmatrix}
\end{equation}

For the reflected wave, we have:

\begin{equation}
\begin{pmatrix}
M_{PP} & M_{PSV} & M_{PSH} \\
M_{SVP} & M_{SVSV} & M_{SVSH} \\
M_{SHP} & M_{SHSV} & M_{SHSH}
\end{pmatrix} = \begin{pmatrix}
\sin \alpha & 0 & - \cos \alpha \\
\cos \alpha & 0 & \sin \alpha \\
0 & - 1 & 0
\end{pmatrix} \begin{pmatrix}
M_{RR} & M_{RT} & M_{RZ} \\
M_{TR} & M_{TT} & M_{TZ} \\
M_{ZR} & M_{ZT} & M_{ZZ}
\end{pmatrix} \begin{pmatrix}
\sin \alpha & \cos \alpha & 0 \\
0 & 0 & - 1 \\
- \cos \alpha & \sin \alpha & 0
\end{pmatrix}
\end{equation}

In the $(\vec{e}_P, \vec{e}_{SV}, \vec{e}_{SH})$ coordinate system, we have:

\begin{equation}
\vec{\Gamma} = \begin{pmatrix}
\gamma_1 \\
\gamma_2 \\
\gamma_3
\end{pmatrix} = \begin{pmatrix}
1 \\
0 \\
0
\end{pmatrix}
\end{equation}

from Equation 9.13.1 of Pujol (2003), we have:

\begin{equation}
\begin{pmatrix}
A_P \\
A_{SV} \\
A_{SH}
\end{pmatrix} =\begin{pmatrix}
M_{PP} \\
M_{SVP} \\
M_{SHP}
\end{pmatrix}
\end{equation}

\section{Getting the reflection, conversion and transmission coefficients}

We compute the reflection, conversion and transmission coefficients at the interface between two homogeneous media, following Aki and Richards (2002, ch. 5.2).

\subsection{SH-wave}

We have $u_x = 0$, $u_z = 0$ and $\frac{\partial}{\partial y} = 0$. Thus the wave equations become:

\begin{equation}
\begin{split}
\rho \frac{\partial ^2 u_y}{\partial t^2} & = \frac{\partial \sigma_{yx}}{\partial x} + \frac{\partial \sigma_{yz}}{\partial z} \\
\sigma_{xy} & = \mu \frac{\partial u_y}{\partial x} \\
\sigma_{yz} & = \mu \frac{\partial u_y}{\partial z} \\
\sigma_{xx} & = \sigma_{yy} = \sigma_{zz} = \sigma_{xz} = 0
\end{split}
\end{equation}

The incident SH-wave is of the form:

\begin{equation}
u_y (t) = \acute S_1 exp (i \omega (t - p_{\beta_1} x - q_{\beta_1} z))
\end{equation}

From Snell's law, we have $\frac{sin j_1}{\beta_1} = \frac{sin j_2}{\beta_2}$

At $z = 0$, we have $u_y (z^+) = u_y (z^-)$, and $\sigma_{yz} (z^+) = \sigma_{yz} (z^-)$ thus:

\begin{equation}
\begin{split}
& \acute S_1 + \grave S_1 = \acute S_2 \\
& - \mu_1 i \omega q_{\beta_1} \acute S_1 + \mu_1 i \omega q_{\beta_1} \grave S_1 = - \mu_2 i \omega q_{\beta_2} \acute S_2
\end{split}
\end{equation}

Therefore, using $q_{\beta_1} = \frac{cos j_1}{\beta_1}$ and $q_{\beta_2} = \frac{cos j_2}{\beta_2}$, we find:

\begin{equation}
\begin{split}
& \grave S_1 = \acute S_1 \frac{\rho_1 \beta_1 cos j_1 - \rho_2 \beta_2 cos j_2}{\rho_1 \beta_1 cos j_1 + \rho_2 \beta_2 cos j_2} \\
& \acute S_2 = \acute S_1 \frac{2 \rho_1 \beta_1 cos j_1}{\rho_1 \beta_1 cos j_1 + \rho_2 \beta_2 cos j_2}
\end{split}
\end{equation}

\subsection{P-wave}

We have $u_y = 0$ and $\frac{\partial}{\partial y} = 0$. Thus the wave equations become:

\begin{equation}
\begin{split}
\rho \frac{\partial ^2 u_x}{\partial t^2} & = \frac{\partial \sigma_{xx}}{\partial x} + \frac{\partial \sigma_{xz}}{\partial z} \\
\rho \frac{\partial ^2 u_z}{\partial t^2} & = \frac{\partial \sigma_{zx}}{\partial x} + \frac{\partial \sigma_{zz}}{\partial z} \\
\sigma_{xx} & = (\lambda + 2 \mu) \frac{\partial u_x}{\partial x} + \lambda \frac{\partial u_z}{\partial z} \\
\sigma_{yy} & = \lambda (\frac{\partial u_x}{\partial x} + \frac{\partial u_z}{\partial z}) \\
\sigma_{zz} & = \lambda \frac{\partial u_x}{\partial x} + (\lambda + 2 \mu) \frac{\partial u_z}{\partial z} \\
\sigma_{xz} & = \mu (\frac{\partial u_x}{\partial z} + \frac{\partial u_z}{\partial x}) \\
\sigma_{xy} & = \sigma_{yz} = 0
\end{split}
\end{equation}

The incident P-wave is of the form:

\begin{equation}
\begin{split}
u_x (t) & = \acute P_1 sin i_1 exp (i \omega (t - p_{\alpha_1} x - q_{\alpha_1} z)) \\
u_z (t) & = \acute P_1 cos i_1 exp (i \omega (t - p_{\alpha_1} x - q_{\alpha_1} z))
\end{split}
\end{equation}

From Snell's law, we have $\frac{sin i_1}{\alpha_1} = \frac{sin i_2}{\alpha_2} = \frac{sin j_1}{\beta_1} = \frac{sin j_2}{\beta_2}$

At $z = 0$, we have $u_x (z^+) = u_x (z^-)$, $u_z (z^+) = u_z (z^-)$, $\sigma_{xz} (z^+) = \sigma_{xz} (z^-)$ and $\sigma_{zz} (z^+) = \sigma_{zz} (z^-)$ thus:

\begin{equation}
\begin{split}
\acute P_1 sin i_1 + \grave P_1 sin i_1 + \grave S_1 cos j_1 = \acute P_2 sin i_2 + \acute S_2 cos j_2 \\
- \acute P_1 cos i_1 + \grave P_1 cos i_1 - \grave S_1 sin j_1 = - \acute P_2 cos i_2 + \acute S_2 sin j_2 \\
\mu_1 (- q{\alpha_1} \acute P_1 sin i_1 + q_{\alpha_1} \grave P_1 sin i_1 + q_{\beta_1} \grave S_1 cos j_1) + \mu_1 (- p_{\alpha_1} \acute P_1 cos i_1 + p_{\alpha_1} \grave P_1 cos i_1 - p_{\beta_1} \grave S_1 sin j_1) = \\
\mu_2 (- q_{\alpha_2} \acute P_2 sin i_2 - q_{\beta_2} \acute S_2 cos j_2) + \mu_2 (- p_{\alpha_2} \acute P_2 cos i_2 + p_{\beta_2} \acute S_2 sin j_2) \\
\lambda_1 (p_{\alpha_1} \acute P_1 sin i_1 + p_{\alpha_1} \grave P_1 sin i_1 + p_{\beta_1} \grave S_1 cos j_1) + (\lambda_1 + 2 \mu_1) (q_{\alpha_1} \acute P_1 cos i_1 + q_{\alpha_1} \grave P_1 cos i_1 - q_{\beta_1} \grave S_1 sin j_1) = \\
\lambda_2 (p_{\alpha_2} \acute P_2 sin i_2 + p_{\beta_2} \acute S_2 cos j_2) + (\lambda_2 + 2 \mu_2) (q_{\alpha_2} \acute P_2 cos i_2 - q_{\beta_2} \acute S_2 sin j_2)
\end{split}
\end{equation}

that is:

\begin{equation}
\begin{split}
- \acute P_2 sin i_2 - \acute S_2 cos j_2 + \grave P_1 sin i_1 + \grave S_1 cos j_1 = - \acute P_1 sin i_1 \\
\acute P_2 cos i_2 - \acute S_2 sin j_2 + \grave P_1 cos i_1 - \grave S_1 sin j_1 = \acute P_1 cos i_1 \\
\mu_2 q_{\alpha_2} \acute P_2 sin i_2 + \mu_2 p_{\alpha_2} \acute P_2 cos i_2 + \mu_2 q_{\beta_2} \acute S_2 cos j_2 - \mu_2 p_{\beta_2} \acute S_2 sin j_2 \\
+ \mu_1 q{\alpha_1} \grave P_1 sin i_1 + \mu_1 p_{\alpha_1} \grave P_1 cos i_1 + \mu_1 q_{\beta_1} \grave S_1 cos j_1 - \mu_1 p_{\beta_1} \grave S_1 sin j_1 = \\
\mu_1 q_{\alpha_1} \acute P_1 sin i_1 + \mu_1 p_{\alpha_1} \acute P_1 cos i_1 \\
- \lambda_2 p_{\alpha_2} \acute P_2 sin i_2 - (\lambda_2 + 2 \mu_2) q_{\alpha_2} \acute P_2 cos i_2 - \lambda_2 p_{\beta_2} \acute S_2 cos j_2 + (\lambda_2 + 2 \mu_2) q_{\beta_2} \acute S_2 sin j_2 \\
+ \lambda_1 p_{\alpha_1} \grave P_1 sin i_1 + (\lambda_1 + 2 \mu_1) q_{\alpha_1} \grave P_1 cos i_1 + \lambda_1 p_{\beta_1} \grave S_1 cos j_1 - (\lambda_1 + 2 \mu_1) q_{\beta_1} \grave S_1 sin j_1 = \\
- \lambda_1 p_{\alpha_1} \acute P_1 sin i_1 - (\lambda_1 + 2 \mu_1) q_{\alpha_1} \acute P_1 cos i_1
\end{split}
\end{equation}

that is:

\begin{equation}
\begin{split}
- \acute P_2 sin i_2 - \acute S_2 cos j_2 + \grave P_1 sin i_1 + \grave S_1 cos j_1 = - \acute P_1 sin i_1 \\
\acute P_2 cos i_2 - \acute S_2 sin j_2 + \grave P_1 cos i_1 - \grave S_1 sin j_1 = \acute P_1 cos i_1 \\
(\mu_2 q_{\alpha_2} sin i_2 + \mu_2 p_{\alpha_2} cos i_2) \acute P_2 + (\mu_2 q_{\beta_2} cos j_2 - \mu_2 p_{\beta_2} sin j_2) \acute S_2 \\
+ (\mu_1 q_{\alpha_1} sin i_1 + \mu_1 p_{\alpha_1} cos i_1) \grave P_1 + (\mu_1 q_{\beta_1} cos j_1 - \mu_1 p_{\beta_1} sin j_1) \grave S_1 = \\
(\mu_1 q_{\alpha_1} sin i_1 + \mu_1 p_{\alpha_1} cos i_1) \acute P_1 \\
- (\lambda_2 p_{\alpha_2} sin i_2 + (\lambda_2 + 2 \mu_2) q_{\alpha_2} cos i_2) \acute P_2 - (\lambda_2 p_{\beta_2} cos j_2 - (\lambda_2 + 2 \mu_2) q_{\beta_2} sin j_2) \acute S_2 \\
+ (\lambda_1 p_{\alpha_1} sin i_1 + (\lambda_1 + 2 \mu_1) q_{\alpha_1} cos i_1) \grave P_1 + (\lambda_1 p_{\beta_1} cos j_1 - (\lambda_1 + 2 \mu_1) q_{\beta_1} sin j_1) \grave S_1 = \\
- (\lambda_1 p_{\alpha_1} sin i_1 + (\lambda_1 + 2 \mu_1) q_{\alpha_1} cos i_1) \acute P_1
\end{split}
\end{equation}

that is:

\begin{equation}
\begin{split}
- \alpha_2 p_{\alpha_2} \acute P_2 - cos j_2 \acute S_2 + \alpha_1 p_{\alpha_1} \grave P_1 + cos j_1 \grave S_1 = - \alpha_1 p_{\alpha_1} \acute P_1 \\
cos i_2 \acute P_2 - \beta_2 p_{\beta_2} \acute S_2 + cos i_1 \grave P_1 - \beta_1 p_{\beta_1} \grave S_1 = cos i_1 \acute P_1 \\
(2 \rho_2 \beta_2^2 p_{\alpha_2} cos i_2) \acute P_2 + \rho_2 \beta_2 (1 - 2 \beta_2^2 p_{\beta_2}^2) \acute S_2 \\
+ (2 \rho_1 \beta_1^2 p_{\alpha_1} cos i_1) \grave P_1 + \rho_1 \beta_1 (1 - 2 \beta_1^2 p_{\beta_1}^2) \grave S_1 = \\
(2 \rho_1 \beta_1^2 p_{\alpha_1} cos i_1) \acute P_1 \\
- \rho_2 \alpha_2 (1 - 2 \beta_2^2 p_{\alpha_2}^2) \acute P_2 + 2 \rho_2 \beta_2^2 p_{\beta_2} cos j_2 \acute S_2 \\
+ \rho_1 \alpha_1 (1 - 2 \beta_1^2 p_{\alpha_1}^2) \grave P_1 - 2 \rho_1 \beta_1^2 p_{\beta_1} cos j_1 \grave S_1 = \\
- \rho_1 \alpha_1 (1 - 2 \beta_1^2 p_{\alpha_1}^2) \acute P_1
\end{split}
\end{equation}

that is:

\begin{equation}
M \begin{pmatrix}
\acute P_2 \\
\acute S_2 \\
\grave P_1 \\
\grave S_1
\end{pmatrix} = N \begin{pmatrix}
0 \\
0 \\
\acute P_1 \\
0
\end{pmatrix}
\end{equation}

with:

\begin{equation}
M = \begin{pmatrix}
- \alpha_2 p_{\alpha_2} & - cos j_2 & \alpha_1 p_{\alpha_1} & cos j_1 \\
cos i_2 & - \beta_2 p_{\beta_2} & cos i_1 & - \beta_1 p_{\beta_1} \\
2 \rho_2 \beta_2^2 p_{\alpha_2} cos i_2 & \rho_2 \beta_2 (1 - 2 \beta_2^2 p_{\beta_2}^2) & 2 \rho_1 \beta_1^2 p_{\alpha_1} cos i_1 & \rho_1 \beta_1 (1 - 2 \beta_1^2 p_{\beta_1}^2) \\
- \rho_2 \alpha_2 (1 - 2 \beta_2^2 p_{\alpha_2}^2) & 2 \rho_2 \beta_2^2 p_{\beta_2} cos j_2 & \rho_1 \alpha_1 (1 - 2 \beta_1^2 p_{\alpha_1}^2) & - 2 \rho_1 \beta_1^2 p_{\beta_1} cos j_1
\end{pmatrix}
\end{equation}

and:

\begin{equation}
N = \begin{pmatrix}
0 & 0 & - \alpha_1 p_{\alpha_1} & 0 \\
0 & 0 & cos i_1 & 0 \\
0 & 0 & 2 \rho_1 \beta_1^2 p_{\alpha_1} cos i_1 & 0 \\
0 & 0 & - \rho_1 \alpha_1 (1 - 2 \beta_1^2 p_{\alpha_1}^2) & 0
\end{pmatrix}
\end{equation}

\subsection{SV-wave}

We have $u_y = 0$ and $\frac{\partial}{\partial y} = 0$. Thus the wave equations become:

\begin{equation}
\begin{split}
\rho \frac{\partial ^2 u_x}{\partial t^2} & = \frac{\partial \sigma_{xx}}{\partial x} + \frac{\partial \sigma_{xz}}{\partial z} \\
\rho \frac{\partial ^2 u_z}{\partial t^2} & = \frac{\partial \sigma_{zx}}{\partial x} + \frac{\partial \sigma_{zz}}{\partial z} \\
\sigma_{xx} & = (\lambda + 2 \mu) \frac{\partial u_x}{\partial x} + \lambda \frac{\partial u_z}{\partial z} \\
\sigma_{yy} & = \lambda (\frac{\partial u_x}{\partial x} + \frac{\partial u_z}{\partial z}) \\
\sigma_{zz} & = \lambda \frac{\partial u_x}{\partial x} + (\lambda + 2 \mu) \frac{\partial u_z}{\partial z} \\
\sigma_{xz} & = \mu (\frac{\partial u_x}{\partial z} + \frac{\partial u_z}{\partial x}) \\
\sigma_{xy} & = \sigma_{yz} = 0
\end{split}
\end{equation}

The incident S-wave is of the form:

\begin{equation}
\begin{split}
u_x (t) & = \acute S_1 cos j_1 exp (i \omega (t - p_{\beta_1} x - q_{\beta_1} z)) \\
u_z (t) & = - \acute S_1 sin j_1 exp (i \omega (t - p_{\beta_1} x - q_{\beta_1} z))
\end{split}
\end{equation}

From Snell's law, we have $\frac{sin i_1}{\alpha_1} = \frac{sin i_2}{\alpha_2} = \frac{sin j_1}{\beta_1} = \frac{sin j_2}{\beta_2}$

At $z = 0$, we have $u_x (z^+) = u_x (z^-)$, $u_z (z^+) = u_z (z^-)$, $\sigma_{xz} (z^+) = \sigma_{xz} (z^-)$ and $\sigma_{zz} (z^+) = \sigma_{zz} (z^-)$ thus:

\begin{equation}
\begin{split}
\acute S_1 cos j_1 + \grave P_1 sin i_1 + \grave S_1 cos j_1 = \acute P_2 sin i_2 + \acute S_2 cos j_2 \\
\acute S_1 sin j_1 + \grave P_1 cos i_1 - \grave S_1 sin j_1 = - \acute P_2 cos i_2 + \acute S_2 sin j_2 \\
\mu_1 (- q_{\beta_1} \acute S_1 cos j_1 + q_{\alpha_1} \grave P_1 sin i_1 + q_{\beta_1} \grave S_1 cos j_1) + \mu_1 (p_{\beta_1} \acute S_1 sin j_1 + p_{\alpha_1} \grave P_1 cos i_1 - p_{\beta_1} \grave S_1 sin j_1) = \\
\mu_2 (- q_{\alpha_2} \acute P_2 sin i_2 - q_{\beta_2} \acute S_2 cos j_2) + \mu_2 (- p_{\alpha_2} \acute P_2 cos i_2 + p_{\beta_2} \acute S_2 sin j_2) \\
\lambda_1 (p_{\beta_1} \acute S_1 cos j_1 + p_{\alpha_1} \grave P_1 sin i_1 + p_{\beta_1} \grave S_1 cos j_1) + (\lambda_1 + 2 \mu_1) (- q_{\beta_1} \acute S_1 sin j_1 + q_{\alpha_1} \grave P_1 cos i_1 - q_{\beta_1} \grave S_1 sin j_1) = \\
\lambda_2 (p_{\alpha_2} \acute P_2 sin i_2 + p_{\beta_2} \acute S_2 cos j_2) + (\lambda_2 + 2 \mu_2) (q_{\alpha_2} \acute P_2 cos i_2 - q_{\beta_2} \acute S_2 sin j_2)
\end{split}
\end{equation}

that is:

\begin{equation}
\begin{split}
- \acute P_2 sin i_2 - \acute S_2 cos j_2 + \grave P_1 sin i_1 + \grave S_1 cos j_1 = - \acute S_1 cos j_1 \\
\acute P_2 cos i_2 - \acute S_2 sin j_2 + \grave P_1 cos i_1 - \grave S_1 sin j_1 = - \acute S_1 sin j_1 \\
\mu_2 q_{\alpha_2} \acute P_2 sin i_2 + \mu_2 p_{\alpha_2} \acute P_2 cos i_2 + \mu_2 q_{\beta_2} \acute S_2 cos j_2 - \mu_2 p_{\beta_2} \acute S_2 sin j_2 \\
+ \mu_1 q{\alpha_1} \grave P_1 sin i_1 + \mu_1 p_{\alpha_1} \grave P_1 cos i_1 + \mu_1 q_{\beta_1} \grave S_1 cos j_1 - \mu_1 p_{\beta_1} \grave S_1 sin j_1 = \\
\mu_1 q_{\beta_1} \acute S_1 cos j_1 - \mu_1 p_{\beta_1} \acute S_1 sin j_1 \\
- \lambda_2 p_{\alpha_2} \acute P_2 sin i_2 - (\lambda_2 + 2 \mu_2) q_{\alpha_2} \acute P_2 cos i_2 - \lambda_2 p_{\beta_2} \acute S_2 cos j_2 + (\lambda_2 + 2 \mu_2) q_{\beta_2} \acute S_2 sin j_2 \\
+ \lambda_1 p_{\alpha_1} \grave P_1 sin i_1 + (\lambda_1 + 2 \mu_1) q_{\alpha_1} \grave P_1 cos i_1 + \lambda_1 p_{\beta_1} \grave S_1 cos j_1 - (\lambda_1 + 2 \mu_1) q_{\beta_1} \grave S_1 sin j_1 = \\
- \lambda_1 p_{\beta_1} \acute S_1 cos j_1 + (\lambda_1 + 2 \mu_1) q_{\beta_1} \acute S_1 sin j_1
\end{split}
\end{equation}

that is:

\begin{equation}
\begin{split}
- \acute P_2 sin i_2 - \acute S_2 cos j_2 + \grave P_1 sin i_1 + \grave S_1 cos j_1 = - \acute S_1 cos j_1 \\
\acute P_2 cos i_2 - \acute S_2 sin j_2 + \grave P_1 cos i_1 - \grave S_1 sin j_1 = - \acute S_1 sin j_1 \\
(\mu_2 q_{\alpha_2} sin i_2 + \mu_2 p_{\alpha_2} cos i_2) \acute P_2 + (\mu_2 q_{\beta_2} cos j_2 - \mu_2 p_{\beta_2} sin j_2) \acute S_2 \\
+ (\mu_1 q_{\alpha_1} sin i_1 + \mu_1 p_{\alpha_1} cos i_1) \grave P_1 + (\mu_1 q_{\beta_1} cos j_1 - \mu_1 p_{\beta_1} sin j_1) \grave S_1 = \\
(\mu_1 q_{\beta_1} cos j_1 - \mu_1 p_{\beta_1} sin j_1) \acute S_1 \\
- (\lambda_2 p_{\alpha_2} sin i_2 + (\lambda_2 + 2 \mu_2) q_{\alpha_2} cos i_2) \acute P_2 - (\lambda_2 p_{\beta_2} cos j_2 - (\lambda_2 + 2 \mu_2) q_{\beta_2} sin j_2) \acute S_2 \\
+ (\lambda_1 p_{\alpha_1} sin i_1 + (\lambda_1 + 2 \mu_1) q_{\alpha_1} cos i_1) \grave P_1 + (\lambda_1 p_{\beta_1} cos j_1 - (\lambda_1 + 2 \mu_1) q_{\beta_1} sin j_1) \grave S_1 = \\
(- \lambda_1 p_{\beta_1} cos j_1 + (\lambda_1 + 2 \mu_1) q_{\beta_1} sin j_1) \acute S_1
\end{split}
\end{equation}

that is:

\begin{equation}
\begin{split}
- \alpha_2 p_{\alpha_2} \acute P_2 - cos j_2 \acute S_2 + \alpha_1 p_{\alpha_1} \grave P_1 + cos j_1 \grave S_1 = - cos j_1 \acute S_1 \\
cos i_2 \acute P_2 - \beta_2 p_{\beta_2} \acute S_2 + cos i_1 \grave P_1 - \beta_1 p_{\beta_1} \grave S_1 = - \beta_1 p_{\beta_1} \acute S_1 \\
(2 \rho_2 \beta_2^2 p_{\alpha_2} cos i_2) \acute P_2 + \rho_2 \beta_2 (1 - 2 \beta_2^2 p_{\beta_2}^2) \acute S_2 \\
+ (2 \rho_1 \beta_1^2 p_{\alpha_1} cos i_1) \grave P_1 + \rho_1 \beta_1 (1 - 2 \beta_1^2 p_{\beta_1}^2) \grave S_1 = \\
\rho_1 \beta_1 (1 - 2 \beta_1^2 p_{\beta_1}^2) \acute S_1 \\
- \rho_2 \alpha_2 (1 - 2 \beta_2^2 p_{\alpha_2}^2) \acute P_2 + 2 \rho_2 \beta_2^2 p_{\beta_2} cos j_2 \acute S_2 \\
+ \rho_1 \alpha_1 (1 - 2 \beta_1^2 p_{\alpha_1}^2) \grave P_1 - 2 \rho_1 \beta_1^2 p_{\beta_1} cos j_1 \grave S_1 = \\
2 \rho_1 \beta_1^2 p_{\beta_1} cos j_1 \acute S_1
\end{split}
\end{equation}

that is:

\begin{equation}
M \begin{pmatrix}
\acute P_2 \\
\acute S_2 \\
\grave P_1 \\
\grave S_1
\end{pmatrix} = N \begin{pmatrix}
0 \\
0 \\
0 \\
\acute S_1
\end{pmatrix}
\end{equation}

with:

\begin{equation}
M = \begin{pmatrix}
- \alpha_2 p_{\alpha_2} & - cos j_2 & \alpha_1 p_{\alpha_1} & cos j_1 \\
cos i_2 & - \beta_2 p_{\beta_2} & cos i_1 & - \beta_1 p_{\beta_1} \\
2 \rho_2 \beta_2^2 p_{\alpha_2} cos i_2 & \rho_2 \beta_2 (1 - 2 \beta_2^2 p_{\beta_2}^2) & 2 \rho_1 \beta_1^2 p_{\alpha_1} cos i_1 & \rho_1 \beta_1 (1 - 2 \beta_1^2 p_{\beta_1}^2) \\
- \rho_2 \alpha_2 (1 - 2 \beta_2^2 p_{\alpha_2}^2) & 2 \rho_2 \beta_2^2 p_{\beta_2} cos j_2 & \rho_1 \alpha_1 (1 - 2 \beta_1^2 p_{\alpha_1}^2) & - 2 \rho_1 \beta_1^2 p_{\beta_1} cos j_1
\end{pmatrix}
\end{equation}

and:

\begin{equation}
N = \begin{pmatrix}
0 & 0 & 0 & - cos j_1 \\
0 & 0 & 0 & - \beta_1 p_{\beta_1} \\
0 & 0 & 0 & \rho_1 \beta_1 (1 - 2 \beta_1^2 p_{\beta_1}^2) \\
0 & 0 & 0 & 2 \rho_1 \beta_1^2 p_{\beta_1} cos j_1
\end{pmatrix}
\end{equation}

\chapter{Ray tracing}

We solve the eikonal and the transport equations following \v Cevern\'y (2001, ch. 3.1).

\section{Eikonal equation}

Following \v Cevern\'y (2001, ch. 2.4), the eikonal equation is:

\begin{equation}
\nabla T . \nabla T = \frac{1}{V^2}
\end{equation}

with $V = \alpha$ or $V= \beta$. Using the Hamiltonian, it can also be written as:

\begin{equation}
\mathcal H (x_i, p_i) = \frac{1}{2} (p_i^2 - \frac{1}{V^2}) = 0
\end{equation}

where $p_i = \frac{\partial T}{\partial x_i}$.

We define the auxiliary variable $\sigma$ by:

\begin{equation}
\frac{dx_i}{d\sigma} = \frac{\partial \mathcal H}{\partial p_i} \text{ and } \frac{dp_i}{d\sigma} = - \frac{\partial \mathcal H}{\partial x_i}
\end{equation}

We get:

\begin{equation}
\frac{dT}{d\sigma} = \frac{\partial T}{\partial x_i} \frac{\partial x_i}{\partial \sigma} = p_i \frac{\partial \mathcal H}{\partial p_i} = \frac{1}{V^2}
\end{equation}

thus we have:

\begin{equation}
T = T_0 + \frac{1}{V^2} \sigma \text{ and } \sigma = V^2 (T - T_0)
\end{equation}

\subsection{Constant velocity}

We have:

\begin{equation}
\frac{dp_i}{d\sigma} = - \frac{\partial \mathcal H}{\partial x_i} = \frac{1}{2} \frac{\partial}{\partial x_i} (\frac{1}{V^2}) = - \frac{1}{V^3} \frac{\partial V}{\partial x_i} = 0
\end{equation}

thus:

\begin{equation}
\begin{split}
p_1 & = p_{10} \\
p_2 & = p_{20} \\
p_3 & = p_{30}
\end{split}
\end{equation}

We have:

\begin{equation}
x_i = x_{i0} + \frac{\partial \mathcal H}{\partial p_i} \sigma = x_{i0} + p_i \sigma
\end{equation}

thus:

\begin{equation}
\begin{split}
x_1 & = x_{10} + p_{10} V^2 (T - T_0) \\
x_2 & = x_{20} + p_{20} V^2 (T - T_0) \\
x_3 & = x_{30} + p_{30} V^2 (T - T_0)
\end{split}
\end{equation}

\subsection{Constant gradient of velocity}

We write the velocity as $V = a z + b$.

We have:

\begin{equation}
\frac{dp_1}{d\sigma} = 0 \text{, } \frac{dp_2}{d\sigma} = 0 \text{ and } \frac{dp_3}{d\sigma} = - \frac{1}{V^3} \frac{\partial V}{\partial z} = - \frac{a}{(a z + b)^3}
\end{equation}

thus:

\begin{equation}
\begin{split}
p_1 & = p_{10} \\
p_2 & = p_{20} \\
p_3 & = p_{30} - \frac{a}{(a z + b)^3} \sigma = p_{30} - \frac{a}{a z + b} (T - T_0)
\end{split}
\end{equation}

We have:

\begin{equation}
\begin{split}
x_1 & = x_{10} + p_{10} \sigma \\
x_2 & = x_{20} + p_{20} \sigma \\
x_3 & = x_{30} + p_{30} \sigma - \frac{1}{2} \frac{a}{(a z + b)^3} \sigma^2
\end{split}
\end{equation}

thus:

\begin{equation}
\begin{split}
x_1 & = x_{10} + p_{10} (a z + b)^2 (T - T_0) \\
x_2 & = x_{20} + p_{20} (a z + b)^2 (T - T_0) \\
x_3 & = x_{30} + p_{30} (a z + b)^2 (T - T_0) - \frac{1}{2} a (a z + b) (T - T_0)^2
\end{split}
\end{equation}

\section{Transport equation}

Following \v Cevern\'y (2001, ch. 2.4), the transport equation is:

\begin{equation}
2 \nabla T . \nabla (\sqrt{\rho V^2} A) + \sqrt{\rho V^2} A \nabla^2 T = 0
\end{equation}

with $V = \alpha$ or $V= \beta$ and $A$ is the amplitude of the P-wave or one of the two components of the S-wave.

\subsection{Constant velocity}

We have $(\nabla T)_i = p_i = p_{i0}$ thus $\nabla^2 T = 0$ and the wave equation becomes:

\begin{equation}
2 \nabla T . \nabla (\sqrt{\rho V^2} A) = 0
\end{equation}

As $\rho$ and $V$ are constant, we get:

\begin{equation}
\nabla T . \nabla A = p_i \frac{\partial A}{\partial x_i} = 0
\end{equation}

However, we have:

\begin{equation}
\frac{\partial A}{\partial \sigma} = \frac{\partial A}{\partial x_i} \frac{\partial x_i}{\partial \sigma} = \frac{\partial A}{\partial x_i} \frac{\partial \mathcal H}{\partial p_i} = \frac{\partial A}{\partial x_i} p_i
\end{equation}

Thus:

\begin{equation}
\frac{\partial A}{\partial \sigma} = 0 \text{ that is } A = A_0
\end{equation}

\subsection{Constant gradient of velocity}

We have:

\begin{equation}
\nabla^2 T = \frac{a^2}{(a z + b)^2} (T - T_0)
\end{equation}

If we assume constant density, we get the transport equation:

\begin{equation}
2 A \nabla T . \nabla V + 2 V \nabla T . \nabla A + A \frac{a^2}{a z + b} (T - T_0) = 0
\end{equation}

\end{document}
