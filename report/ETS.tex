\documentclass[main.tex]{subfiles}
 
\begin{document}

\part{Episodic Tremor and Slip (ETS)}

\chapter{Slow slip}

Slow slip on the plate boundary is inferred to happen when there is a reversal of the direction of motion at GPS stations, compared to the secular motion of the surface displacement. \\

The amplitude of the horizontal displacement measured by the GPS stations at the surface is a few millimetres. Dragert \textit{et al.} (2001 \cite{DRA_2001}) found displacements ranging from 2 to 4 millimetres. Dragert \textit{et al.} (2004 \cite{DRA_2004}) found an average displacement of 5 millimetres. Szeliga \textit{et al.} (2008 \cite{SZE_2008}) found a displacement consistently lower than 6 millimetres. This should be compared to a secular velocity of 5.6 millimetres per year on average, and an inter-slip velocity of 9.7 millimetres per year on average (Dragert \textit{et al.}, 2004 \cite{DRA_2004}). \\

The reversal of the direction of motion is observed during a few weeks at each GPS station. Dragert \textit{et al} (2001 \cite{DRA_2001}) observed a reversal lasting about 6 to 15 days depending on the GPS station for the summer 1999 event. Miller \textit{et al.} (2002 \cite{MIL_2002}) observed a reversal lasting on average 2 to 4 weeks for the eight events between 1992 and 2001. Dragert \textit{et al.} (2004 \cite{DRA_2004}) observed reversals lasting 1 to 3 weeks. The average dislocation risetime was found to be 14 days with a maximum of about 30 days (Schmidt and Gao, 2010 \cite{SCH_2010}).\\
 
The reversal of the direction of motion does not occur at the same time for each GPS station. Dragert \textit{et al.} (2001 \cite{DRA_2001}) observed a 35 days time lag between the beginning of the reverse displacement at the most southeastern station and beginning of the reverse displacement at the most northwestern station for the summer 1999 event. Miller \textit{et al.} (2002 \cite{MIL_2002}) observed an average time lag of 3 weeks between the beginning of the event at the first station and the beginning of the event at the last station for the eight events between 1992 and 2001. The overall duration of an event is 2 to 7 weeks (Gao \textit{et al.}, 2012 \cite{GAO_2012}). This corresponds to a propagation velocity along the strike of the plate boundary of about 6 kilometres per day for the summer 1999 event (Dragert \textit{et al.}, 2001 \cite{DRA_2001}). Dragert \textit{et al.} (2004 \cite{DRA_2004} found a propagation velocity varying from 5 to 15 kilometres per day. Schmidt and Gao (2010 \cite{SCH_2010}) found an average propagation rate for the slip initiation of 5.9 kilometres per day, although some fault elements showed a rate as high as 17 kilometres per day. \\

The recurrence interval of the eight slow slip events between 1992 and 2001 was on average 14.5 months according to Miller \textit{et al.} (2002 \cite{MIL_2002}), or between 13 and 16 months according to Dragert \textit{et al.} (2004 \cite{DRA_2004}). \\

Numerical simulation of faulting in an elastic half-space have been carried out by several authors in order to retrieve the corresponding slip at the plate interface. Dragert \textit{et al.} (2001 \cite{DRA_2001}) found a slip of about 2 centimetres between 30 and 40 kilometres depth, and a smaller slip updip of 30 kilometres for the summer 1999 event. Numerical modelling carried out by Miller \textit{et al.} (2002, \cite{MIL_2002}) suggests that the eight events from 1992 to 2001 were evidence of a  creep of a few centimetres along the plate interface at depths of 30 to 50 kilometres. Dragert \textit{et al.} (2004 \cite{DRA_2004}) found a slip of 2 to 4 centimetres on the plate interface bounded by the 25 and 45 kilometres depth contours. Melbourne \textit{et al.} (2005 \cite{MEL_2005}) found a maximum slip of 3.8 centimetres centered at 28 kilometres depth with most of the slip located above 38 kilometres depth. Szeliga \textit{et al.} (2008 \cite{SZE_2008}) found an average slip of 2 to 3 centimetres. The total area where this reversal of the direction of motion was observed was about 50 * 300 kilometres for the summer 1999 event (Dragert \textit{et al.}, 2001 \cite{DRA_2001}). Dragert \textit{et al.} (2001 \cite{DRA_2001}) found that the surface displacement was largest at the sites located more than 100 kilometres landward of the locked zone. Wech and Creager (2008 \cite{WEC_2008}) observed that the western boundary of the area where reversal of the direction of motion occurs is located 75 km east of the downdip edge of the seismogenic zone. The strain release from slow slip was not uniform along strike, and the greater amount of slip is centered around Port Angeles (Schmidt and Gao, 2010 \cite{SCH_2010}).\\

These values of slip and area correspond to earthquakes of moment magnitude 6.7 for the summer 1999 event (Dragert \textit{et al.}, 2001 \cite{DRA_2001}), 6.8 for the July 1998 event, 6.7 for the August 1999 event, 6.7 for the December 2000 event, 6.5 for the February 2002 event (Dragert \textit{et al.}, 2004 \cite{DRA_2004}), 6.6 for the February 2003 event (Melbourne \textit{et al.}, 2005 \cite{MEL_2005}), 6.3 to 6.8 for the events studied by Szeliga \textit{et al.} (2008 \cite{SZE_2008}), and 6.1 to 6.7 for the events studied by Schmidt and Gao (2010 \cite{SCH_2010}). \\

The average stress drop is about 0.01 to 0.10 MPa (Schmidt and Gao, 2010 \cite{SCH_2010}). 

\chapter{Tremor}

The predominant frequency of tremors ranges from 1 to 10 Hz and is lower than that of ordinary earthquakes of similar size (10 to 20 Hz). The envelopes of tremors have gradual rise times and differ from those of a normal earthquake, which has a spike-like envelope shape (Obara, 2002 \cite{OBA_2002}). The frequency content is mainly between 1 and 5 Hz, whereas most of the energy in small earthquakes is above 10 Hz. A tremor onset is usually emergent and the signal consists of pulses of energy, often about a minute in duration. A continuous signal may last from a few minutes to several days (Rogers and Dragert, 2003 \cite{ROG_2003}).

It is only when a number of seismograph signals are viewed together that the similarity in the envelope of the seismic signal at each site identifies the signal as ETS (Rogers and Dragert, 2003 \cite{ROG_2003})

Characteristics: low amplitude, lack of energy at high frequency, emergent onsets, absence of clear impulsive phases (La Rocca \textit{et al.}, 2009)

Depth = 30 km, near the Mohorovi\u ci\'c discontinuity (southwest Japan, Obara, 2002 \cite{OBA_2002}). 20 to 40 km (Rogers and Dragert, 2003 \cite{ROG_2003}).

correlate temporally and spatially with six deep slip events that have occurred over the past 7 years (Rogers and Dragert, 2003 \cite{ROG_2003})

Spatially clustered (Obara \textit{et al., 2004}). Belt-like distribution
Patches tens kilometers of dimension (Ghosh \textit{et al}, 2012).

small-amplitude tremors that lasted from a few minutes to a few days (Obara, 2002 \cite{OBA_2002}). 1 min (Rogers and Dragert, 2003)

Duration of tremor activity = 10 to 20 days in any one region (Rogers and Dragert, 2003 \cite{ROG_2003}). Several days to a few weeks (Obara \textit{et al.}, 2004)

Frequency = rom 1 to 10 Hz (Obara, 2002 \cite{OBA_2002}). Time windows of 35 to 50 min. 1-8 Hz (Ide et al, 2007)
The frequency content is mainly between 1 and 5 Hz, whereas most of the energy in small earthquakes is above 10 Hz (Rogers and Dragert, 2003 \cite{ROG_2003}).

Propagation = along strike 5 to 15 km / day (Rogers and Dragert, 2003 \cite{ROG_2003}). Along-strike 5-17 km / day (Shelly \textit{et al.}, 2007)

Short-trem 15 km up-dip in 20 min (Nankai, Shelly \textit{et al.}, 2007) = 45 km / h

Recurrence interval = 2 to 3 months (eastern Shikoku)

Propagated with a velocity of 4 km/s, that is the source of the tremors was located at a deep portion and the envelopes were propagated by S-wave velocity (Obara, 2002 \cite{OBA_2002}).

no impulsive body wave arrivals $\rightarrow$ Difficult to locate

\chapter{LFEs}

Depth = 30-35 km (nankai, Ide \textit{et al.}, 2007), 7km above regular intraplate earthquakes

Location = spatially clustered, at the plate boundary, 25 to 37 km depth plate boundary contour, between two bands of seismicity (crustal and intraslab earthquakes)

Magnitude= < 2

Mechanism = shear slip on low-angle thrust fault. Point-source, double-couple excitation ; combination of strike-slip and thrust faulting (Bostosck \textit{et al.}, 2012)

Frequency = 1-10 Hz 1-8 Hz (Ide et al, 2007)

\chapter{Slow earthquakes}

Moment / duration : $M_0 = T \times 10^{12-13}$ (slow) versus $M_0 = T^3 \times 10^{15-16}$ (regular)

Moment rate / frequency : $\dot{M}_0 \propto f^{-1}$ (slow) versus $\dot{M}_0 \propto f^{-2}$ (regular)
 
\chapter{Subduction}

Plate convergence (Juan de Fuca) = 4 cm / year
Age 10 million year

up-dip of 45km depth, earthquakes below the reflector (serpentinite dehydration of the mantle), down-dip within subducted crust (basalt-to-eclogite transformation) (Preston \textit{et al.}, 2003) down to 60 km depth

Physical mechanism intraslab earthquakes: dehydration embrittlement, metamorphic dehydration (prograde metamorphism)

Low velocity layer = 3-4km thin, Vs=2-3 km/s Poisson's ratio = 0.4, depth 20-40 km (Nowack and Bostock, 2013).

\end{document}

