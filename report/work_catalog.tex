\documentclass[workdone.tex]{subfiles}
 
\begin{document}

\chapter{Work done on an extended catalog of low-frequency earthquakes}

\section{Description of the dataset}

The template waveforms are the ones obtained by Plourde \textit{et al.} (2015, ~\cite{PLO_2015}). Alexandre Plourde has kindly provided us his template files. \\

The folder \textit{waveforms} contains 91 Matlab files, which contain the template waveforms for the three channels of each of the stations. The folder \textit{detections} contains 89 files, which contain the names of the stations that were used by the network matched filter, and the time of each LFE detection. 66 of these templates have then been grouped into 34 families. The names of the templates in each family are given in the file \textit{family\_list.m}. The locations of the hypocentre for each template are given in the file \textit{template\_locations.txt}. The locations of the hypocentre for each family are given in the file \textit{unique\_families\_NCAL.txt}.

\section{Websites to access the data}

The waveforms from the FAME experiment can be accessed from the IRIS DMC using the Python package obspy, and the subpackage obspy.clients.fdsn. The network code is XQ, and the stations names are ME01 to ME93. \\

The waveforms for the permanent stations of the Northern California Seismic Network can be downloaded from the website of the Northern California Earthquake Data Center (\href{http://ncedc.org/}{NCEDC}). The stations are B039 from the network PB, KCPB, KHBB, KRMB, and KSXB from the network NC, and WDC and YBH from the network BK. Queries for downloading the data in the miniSEED format must be formated as explained here: \\

\href{http://service.ncedc.org/fdsnws/dataselect/1/#description-box}{FDSN Dataselect} \\

The instrument response can be obtained from here: \\

\href{http://service.ncedc.org/fdsnws/station/1/#description-box}{FDSN Station}

\section{Work done}

\paragraph{get\_data.py} This module contains functions to download seismic waveforms from the IRIS Data Management Center (DMC) for the stations of the FAME experiment, or from the Northern California Earthquake Data Center (NCEDC) for the permanent stations. Somehow, waveforms for some of the permanent stations (e.g. station B039 from the Plate Boundary Observatory) could not be downloaded from the NCEDC, but they are available on the IRIS DMC.

\paragraph{get\_waveforms.py} We look at a given LFE family from the catalog of Plourde \textit{et al.} (2015, ~\cite{PLO_2015}). For this family, we know the timing of each LFE. We download the one-minute-long seismic waveform corresponding to each LFE. Then, for each seismic station and each channel, we detrend the data, taper the first and last 5 seconds of the data with a Hann window, remove the instrument response, bandpass filter between 1.5 and 9 Hz, and resample the data to 40 Hz (that is the same sampling rate as the templates given by Plourde). All these preprocessing operations are done with the Python package obspy. We then stack linearly all the waveforms to obtain a waveform template for each station and each channel, and we compare the final templates with the templates from Plourde \textit{et al.} (2015, ~\cite{PLO_2015}). The two templates are similar, but their amplitude is different because we normalized each LFE waveform by the root mean square (RMS) of the waveform, whereas Plourde \textit{et al.} (2015 ~\cite{PLO_2015}) have used another normalization method.

\paragraph{get\_cc\_window.py} We are looking for a systematic way to select a time window in order to get a shorter LFE template, instead of using the full length of the one-minute time window. For that, we need the origin time of the all the LFEs for each family, and the seismic wave travel time from the LFE source to the station for each seismic station. For each family and each station, we assume that the arrival time of the seismic waves is the time where the amplitude of the template waveform is maximum. We first look at a given family. We get the arrival times as a function of the distance from the source of the LFE family to the station for several seismic stations. From that, we can get the origin time of the LFEs for this family. We carry out this procedure for each LFE family. Then we look at a given seismic station, and we get the travel time from the source of the LFE family to this station as a function of the distance from source to station for several LFE families. From this, we compute the slowness of the seismic waves associated to this station. Now that we have an origin time for each LFE of each LFE family, and a seismic wave slowness associated to each station, we can compute the arrival time of the seismic waves at all the stations for all the LFEs of all the LFE families. To compute the templates, we will pick a ten-second-long time window centered on the arrival times.

\paragraph{compute\_templates.py} We look at a given LFE family from the catalog of Plourde \textit{et al.} (2015, ~\cite{PLO_2015}). For this family, we know the timing of each LFE. We download the one-minute-long seismic waveform corresponding to each LFE. Then, for each seismic station and each channel, we detrend the data, taper the first and last 5 seconds of the data with a Hann window, remove the instrument response, bandpass filter between 1.5 and 9 Hz, and resample the data to 20 Hz. All these preprocessing operations are done with the Python package obspy. We then stack linearly all the waveforms to obtain a waveform template for each station and each channel. WE now want to select the best LFEs in order to get a better template. Using the origin times of the LFE, and the seismic wave slowness associated to each station, computed with get\_cc\_window.py, we determine the ten-second-long time window where we expect to see the arrival of the seismic waves during an LFE event. For each LFE, we cross correlate the shorter waveform with the template computed with all LFEs. We then compute a new template using only the LFEs associated with the higher cross correlation values.

\paragraph{get\_stations.py} We look at all the stations from the BK, NC and PB permanent network, and keep the stations that are less than 100 km from the epicenter of the LFE family. For each family, we get the stations where we hope to see an LFE in the seismic waveforms.

\paragraph{save\_waveforms.py} We download the waveform for each LFE and each permanent station less than 100 km from the epicenter of the LFE family. We save all the waveforms into a file for future data analysis. NOT USED ANYMORE

\paragraph{draw\_waveforms.py} We plot the waveform with or without preprocessing (MODWT or wavelet-based-denoising). NOT USED ANYMORE

\paragraph{compute\_stats.py} Compute histogram of cross-correlation of each waveform with the template for each station and each channel.

\paragraph{find\_LFEs.py} We download the seismic waveforms for each station and each channel for the whole time period when we look for LFEs. Then, for each seismic station and each channel, we detrend the data, taper the first and last 5 seconds of the data with a Hann window, remove the instrument response, bandpass filter between 1.5 and 9 Hz, and resample the data to 20 Hz. We then select a time window of the same length as the template. We compute the cross correlation value for each station and each channel. Then we move the selected time window from one sample point, and we compute again the cross correlation. We move the time window sample point by sample point until we have covered the whole time period that we have downloaded. We thus have the stacked cross correlation in function of time for the whole time period. We compute the median absolute deviation (MAD) of the stacked cross correlation function, and whenever the cross correlation is higher that eight times the MAD, we decide that there is an LFE. We marge the LFEs that are within one second of each other, as they are probably the same LFE. We thus obtain a new catalog of LFEs.

\paragraph{compare\_catalog.py} For a given LFE family, we look for LFEs in the time period covered by the catalog from Plourde \textit{et al.} (2015 ~\cite{PLO_2015}). We then count the number of LFEs that are added i our catalog, but are nor present in Plourde's catalog, the number of LFEs that are missing in our catalog, but are present in Plourde's catalog, and the number of LFEs that are present in both catalogs. We compare the cross correlation values associated with the LFEs added in our catalog, with the cross correlation values associated with the LFEs that are present in both catalogs. We note that we only added LFEs associated with a small cross correlation value. We also look at the time lag between the LFEs missing in our catalog, and the closest LFEs. However, this time lag is usually higher than a few seconds, and the fact that some LFEs present in Plourde's catalog are missing in our catalog cannot be explained by the fact that two high cross correlation values could be associated with two different LFEs instead of a single one.

\paragraph{compute\_new\_templates.py}

We first try to find LFEs for family 080421.14.048 between April 21st and April 28th 2008, that is the start date and end date of the catalog obtained by Plourde for this family. We choose different values for the lowest time lag between two LFEs for which we assume that they actually are two different LFEs. We count the number of LFEs missing in our catalog, and the number of LFEs added in our catalog.

timelag Plourde ours missing added both
4s 225 1487 65 1327 160
3s 225 1732 54 1561 171
2s 225 2121 36 1932 189
1s 225 3048 16 2839 209

\section{Things to do}

Look at the maximum of the envelope (instead of the maximum of the raw signal) to find the time of the LFE.

\end{document}

