\documentclass[main.tex]{subfiles}
 
\begin{document}

\part{Data}

\chapter{Time lags}

\section{Description of dataset}

Bla bla about Ghosh et al. (2010)

\section{Websites to access the data}

\href{http://www.fdsn.org/networks/}{\textbf{International Federation of Digital Seismograph Networks}} \\

This website gives a list of the network codes, and the corresponding map, with the names and locations of stations. The selected experiments are:
\begin{itemize}
	\item XG (2009-2011): Cascadia Array of Arrays
	\item XU (2006-2012): Collaborative Research: Earthscope integrated investigations of Cascadia subduction zone tremor, structure and process
\end{itemize}

The stations are the following:
\begin{itemize}
	\item Port Angeles XG - PA01 to PA13
	\item Danz Ranch XG - DR01 to DR10, and DR12
	\item Lost Cause XG - LC01 to LC14
	\item Three Bumps XG - TB01 toTB14
	\item Burnt Hill XG - BH01 to BH11
	\item Cat Lake XG - CL01 to CL20
	\item Gold Creek XG - GC01 to GC14
	\item Blyn XU - BS01 to BS06, BS11, BS20 to BS27
\end{itemize}

\href{http://ds.iris.edu/mda}{\textbf{IRIS DMC MetaData Aggregator}} \\

This website gives for each station:
\begin{itemize}
	\item Location and time of recording
	\item Epoch (effective periods of recording during the time that the station was installed)
	\item Type of instrument
	\item Channels
\end{itemize}

\href{https://www.pnsn.org/tremor}{\textbf{Pacific Northwest Seismic Network tremor catalog}} \\

This website gives the dates and locations of tremor activity in Cascadia. Following Ghosh \textit{et al.} (2012), the selected periods of tremors are:
\begin{itemize}
	\item From November 9th 2009 to November 13th 2009,
	\item From March 16th 2010 to March 19th 2010,
	\item From August 16th 2010 to August 20th 2010.
\end{itemize}

\chapter{LFE catalog}

\section{Description of dataset}

Bla bla about Plourde et al. (2015)

\section{Websites to access the data}

\chapter{Slow slip}

Bla bla about PANGA

\end{document}
