\documentclass[main.tex]{subfiles}
 
\begin{document}

\part{Long-range dependence in low-frequency earthquake catalogs}

\chapter{Data}

\section{LFE catalogs}

\paragraph{The Chestler catalog}

\paragraph{The Plourde catalog} The catalog of Plourde \textit{et al.} (2015 ~\cite{PLO_2015}) contains event times for 66 templates corresponding to 34 families located in Southern Cascadia. They were established using the seismic stations from the EarthScope Flexible Array Mendocino Experiment (FAME) array, and seven permanent stations from the Northern California Seismic Network. The period covered starts on March 21st 2008, and ends on April 30th 2008, which corresponds to the main 2008 ETS event in southern Cascadia. Unfortunately, we do not have (yet) the event times for the three LFE families located on the Maacama and Bucknell Creek faults, which are part of the San Andreas fault zone.

\paragraph{The Shelly catalog}

\paragraph{The Sweet catalog}

\chapter{Method}

\chapter{Results}

\chapter{Discussion and things to do}

In David Shelly's LFE catalog, do we have a lower rate of LFEs when the number of stations decreases (at the end of the dataset)? \\

Can we find a daily periodicity in the LFE catalog because of cultural noise? \\

The aggregation of short-range dependence time series can lead to a long-range dependence time series (see Jan Beran's book page 14, and a paper by Granger). Could an ETAS model with correlation between asperities lead to a model with long-range dependence?

\end{document}
