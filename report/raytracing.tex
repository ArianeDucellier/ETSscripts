\documentclass[workdone.tex]{subfiles}
 
\begin{document}

\section{Ray tracing}

We solve the eikonal and the transport equations following \v Cevern\'y (2001, ch. 3.1).

\subsection{Eikonal equation}

Following \v Cevern\'y (2001, ch. 2.4), the eikonal equation is:

\begin{equation}
\nabla T . \nabla T = \frac{1}{V^2}
\end{equation}

with $V = \alpha$ or $V= \beta$. Using the Hamiltonian, it can also be written as:

\begin{equation}
\mathcal H (x_i, p_i) = \frac{1}{2} (p_i^2 - \frac{1}{V^2}) = 0
\end{equation}

where $p_i = \frac{\partial T}{\partial x_i}$.

We define the auxiliary variable $\sigma$ by:

\begin{equation}
\frac{dx_i}{d\sigma} = \frac{\partial \mathcal H}{\partial p_i} \text{ and } \frac{dp_i}{d\sigma} = - \frac{\partial \mathcal H}{\partial x_i}
\end{equation}

We get:

\begin{equation}
\frac{dT}{d\sigma} = \frac{\partial T}{\partial x_i} \frac{\partial x_i}{\partial \sigma} = p_i \frac{\partial \mathcal H}{\partial p_i} = \frac{1}{V^2}
\end{equation}

thus we have:

\begin{equation}
T = T_0 + \frac{1}{V^2} \sigma \text{ and } \sigma = V^2 (T - T_0)
\end{equation}

\subsubsection{Constant velocity}

We have:

\begin{equation}
\frac{dp_i}{d\sigma} = - \frac{\partial \mathcal H}{\partial x_i} = \frac{1}{2} \frac{\partial}{\partial x_i} (\frac{1}{V^2}) = - \frac{1}{V^3} \frac{\partial V}{\partial x_i} = 0
\end{equation}

thus:

\begin{equation}
\begin{split}
p_1 & = p_{10} \\
p_2 & = p_{20} \\
p_3 & = p_{30}
\end{split}
\end{equation}

We have:

\begin{equation}
x_i = x_{i0} + \frac{\partial \mathcal H}{\partial p_i} \sigma = x_{i0} + p_i \sigma
\end{equation}

thus:

\begin{equation}
\begin{split}
x_1 & = x_{10} + p_{10} V^2 (T - T_0) \\
x_2 & = x_{20} + p_{20} V^2 (T - T_0) \\
x_3 & = x_{30} + p_{30} V^2 (T - T_0)
\end{split}
\end{equation}

\subsubsection{Constant gradient of velocity}

We write the velocity as $V = a z + b$.

We have:

\begin{equation}
\frac{dp_1}{d\sigma} = 0 \text{, } \frac{dp_2}{d\sigma} = 0 \text{ and } \frac{dp_3}{d\sigma} = - \frac{1}{V^3} \frac{\partial V}{\partial z} = - \frac{a}{(a z + b)^3}
\end{equation}

thus:

\begin{equation}
\begin{split}
p_1 & = p_{10} \\
p_2 & = p_{20} \\
p_3 & = p_{30} - \frac{a}{(a z + b)^3} \sigma = p_{30} - \frac{a}{a z + b} (T - T_0)
\end{split}
\end{equation}

We have:

\begin{equation}
\begin{split}
x_1 & = x_{10} + p_{10} \sigma \\
x_2 & = x_{20} + p_{20} \sigma \\
x_3 & = x_{30} + p_{30} \sigma - \frac{1}{2} \frac{a}{(a z + b)^3} \sigma^2
\end{split}
\end{equation}

thus:

\begin{equation}
\begin{split}
x_1 & = x_{10} + p_{10} (a z + b)^2 (T - T_0) \\
x_2 & = x_{20} + p_{20} (a z + b)^2 (T - T_0) \\
x_3 & = x_{30} + p_{30} (a z + b)^2 (T - T_0) - \frac{1}{2} a (a z + b) (T - T_0)^2
\end{split}
\end{equation}

\subsection{Transport equation}

Following \v Cevern\'y (2001, ch. 2.4), the transport equation is:

\begin{equation}
2 \nabla T . \nabla (\sqrt{\rho V^2} A) + \sqrt{\rho V^2} A \nabla^2 T = 0
\end{equation}

with $V = \alpha$ or $V= \beta$ and $A$ is the amplitude of the P-wave or one of the two components of the S-wave.

\subsubsection{Constant velocity}

We have $(\nabla T)_i = p_i = p_{i0}$ thus $\nabla^2 T = 0$ and the wave equation becomes:

\begin{equation}
2 \nabla T . \nabla (\sqrt{\rho V^2} A) = 0
\end{equation}

As $\rho$ and $V$ are constant, we get:

\begin{equation}
\nabla T . \nabla A = p_i \frac{\partial A}{\partial x_i} = 0
\end{equation}

However, we have:

\begin{equation}
\frac{\partial A}{\partial \sigma} = \frac{\partial A}{\partial x_i} \frac{\partial x_i}{\partial \sigma} = \frac{\partial A}{\partial x_i} \frac{\partial \mathcal H}{\partial p_i} = \frac{\partial A}{\partial x_i} p_i
\end{equation}

Thus:

\begin{equation}
\frac{\partial A}{\partial \sigma} = 0 \text{ that is } A = A_0
\end{equation}

\subsubsection{Constant gradient of velocity}

We have:

\begin{equation}
\nabla^2 T = \frac{a^2}{(a z + b)^2} (T - T_0)
\end{equation}

If we assume constant density, we get the transport equation:

\begin{equation}
2 A \nabla T . \nabla V + 2 V \nabla T . \nabla A + A \frac{a^2}{a z + b} (T - T_0) = 0
\end{equation}

\end{document}
