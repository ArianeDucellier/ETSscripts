% !TEX encoding = UTF-8 Unicode
\documentclass{beamer}

\usepackage{amsmath}
\usepackage{color}
\usepackage{gensymb}
\usepackage{hyperref}
\usepackage{textcomp}

\usetheme{Warsaw}

\newcommand{\btVFill}{\vskip0pt plus 1filll}

\title[Data analysis of recordings of slow earthquakes]{Data analysis of recordings of slow earthquakes: Tectonic tremor, low-frequency earthquakes and slow slip events}
\author{Ariane Ducellier}
\institute{University of Washington}
\date{General Exam - October 2019?}

\begin{document}

	\begin{frame}
		\titlepage
	\end{frame}

	%----------------------------------------------------------------------------------------------------------------------------------
	% Introduction
	%----------------------------------------------------------------------------------------------------------------------------------

	\section{Introduction}

	\begin{frame}
		\frametitle{Slow earthquakes}
		% Here image of subduction zone with seismogenic zone  ETS zone / Aseismic slip zone
	\end{frame}

	%----------------------------------------------------------------------------------------------------------------------------------
	% Slow slip
	%----------------------------------------------------------------------------------------------------------------------------------

	\subsection{Sow slip}

	\begin{frame}
		\frametitle{Slow slip}
		% Here image of slow slip adapted for northern cascadia {sesimogenic zone under water, ETS zone under continent)
	\end{frame}

	\begin{frame}
		\frametitle{Slow slip}
		% Here image of GPS data showing slow slip (from Cascadia?)
	\end{frame}

	%----------------------------------------------------------------------------------------------------------------------------------
	% Tectonic tremor
	%----------------------------------------------------------------------------------------------------------------------------------

	\subsection{Tectonic tremor}

	\begin{frame}
		\frametitle{Tectonic tremor}
		\begin{itemize}
			\item Long (several seconds to many minutes)
			\item Low amplitude
			\item Emergent onsets
			\item Absence of clear impulsive phases
		\end{itemize}
	\end{frame}

	\begin{frame}
		\frametitle{Tectonic tremor}
		% Here image of tectonic tremor (signal envelope for 10 stations of seismic array)
	\end{frame}

	%----------------------------------------------------------------------------------------------------------------------------------
	% Low-frequency earthquakes (LFEs)
	%----------------------------------------------------------------------------------------------------------------------------------

	\subsection{Low-frequency earthquakes (LFEs)}

	\begin{frame}
		\frametitle{Low-frequency earthquakes (LFEs)}
		\begin{itemize}
			\item Small magnitude earthquakes (M $\sim$ 1)
			\item Frequency content (1-10 Hz) lower than for ordinary earthquakes (up to 20 Hz)
			\item Source located on the plate boundary,
			\item Focal mechanism: Shear slip on a low-angle thrust fault dipping in the same direction as the plate interface
		\end{itemize}
	\end{frame}

	\begin{frame}
		\frametitle{Low-frequency earthquakes (LFEs)}
		% Here image of low-frequency earthquakes
	\end{frame}

	%----------------------------------------------------------------------------------------------------------------------------------
	% Episodic Tremor and Slip (ETS)
	%----------------------------------------------------------------------------------------------------------------------------------

	\subsection{Episodic Tremor and Slip (ETS)}

	\begin{frame}
		\frametitle{Episodic Tremor and Slip (ETS)}
		\begin{itemize}
			\item Tectonic tremor observations spatially and temporally correlated with slow slip observations (Nankai, Cascadia)
			\item Only biggest tremor episode associated with slow slip
			\item No spatial or temporal correlation in other regions like New Zealand
		\end{itemize}
	\end{frame}

	%----------------------------------------------------------------------------------------------------------------------------------
	% Research questions
	%----------------------------------------------------------------------------------------------------------------------------------
	
	\subsection{Research questions}

	\begin{frame}
		\frametitle{Depth of the source of the tectonic tremor in the eastern Olympic Peninsula}
		\begin{itemize}
			\item Lack of impulsive phases $\rightarrow$ Difficult to determine the depth of the source of the tremor
			\item Tectonic tremor is at least partly made of a swarm of LFEs
			\item LFEs are located on the plate boundary
		\end{itemize}

		\begin{block}{}
			$\rightarrow$ Research question: Is the source of the tectonic tremor located on the plate boundary? What is the depth extent of the location of the source of the tremor?
		\end{block}
	\end{frame}

	\begin{frame}
		\frametitle{A low-frequency earthquake catalog for Northern California}
		\begin{itemize}
			\item LFEs grouped into families of events
			\item All the earthquakes of a given family originate from the same small patch on the plate interface
			\item LFEs recur more or less episodically in a bursty manner
			\item Wide range of recurrence behavior between seismic regions, and within the same seismic region
		\end{itemize}
	\end{frame}

	\begin{frame}
		\frametitle{LFEs in Washington State}
		% Here map of LFE families from Sweet et al. (2019)
	\end{frame}

	\begin{frame}
		\frametitle{LFEs on the San Andreas Fault}
		% Here map of LFE families from Shelly (2017)
	\end{frame}

	\begin{frame}
		\frametitle{A low-frequency earthquakes catalog for southern Cascadia}
		\begin{itemize}
			\item LFE families in southern Cascadia:
			\begin{itemize}
				\item 34 LFE families on the subduction zone
				\item 3 LFE families on two strike-slip faults from the San Andreas Fault system
			\end{itemize}
			\item Wide range of recurrence behavior between Washington State and the San Andreas Fault, and within the San Andreas Fault zone
		\end{itemize}
		
		\begin{block}{}
			$\rightarrow$ Do low-frequency earthquakes families behave similarly or differently in southern Cascadia, compared to Washington State and the San Andreas Fault?
		\end{block}
	\end{frame}

	\begin{frame}
		\frametitle{Detection of slow slip events in New Zealand}
		\begin{itemize}
			\item Small (M $\sim$ 5) or long (several months) slow slip events are harder to detect
			\item In Cascadia, Mexico, tremor used as a proxy to study slow slip events
			\item Different pattern in northern new Zealand:
			\begin{itemize}
				\item Tremor source located downdip of the slow slip on the plate boundary
				\item Tremor activity does not seem to increase during slow slip events
			\end{itemize}
		\end{itemize}

		\begin{block}{}
			$\rightarrow$ Can we detect smaller and / or longer slow slip events in the absence of spatially and temporally correlated tectonic tremor?
		\end{block}			
	\end{frame}

	%----------------------------------------------------------------------------------------------------------------------------------
	% Depth of the source of the tectonic tremor
	%----------------------------------------------------------------------------------------------------------------------------------
				
	\section{Depth of the source of the tectonic tremor}

	\begin{frame}
		\frametitle{Array of Arrays}
	\end{frame}

	%----------------------------------------------------------------------------------------------------------------------------------
	% A low-frequency earthquakes catalog for southern Cascadia
	%----------------------------------------------------------------------------------------------------------------------------------

	\section{A low-frequency earthquakes catalog for southern Cascadia}

	%----------------------------------------------------------------------------------------------------------------------------------
	% Extension of an LFEs catalog for southern Cascadia
	%----------------------------------------------------------------------------------------------------------------------------------

	\subsection{Extension of an LFEs catalog for southern Cascadia}
	
	\begin{frame}
		\frametitle{Current catalog}
		% Here map of LFE families, stations
	\end{frame}

	\begin{frame}
		\frametitle{Current catalog}
		\begin{itemize}
			\item Subduction zone families
			\begin{itemize}
				\item 34 families
				\item Period covered: April 2008
				\item One burst of LFEs lasting a fay days and propagating from south to north
			\end{itemize}
			\item Strike-slip fault families
			\begin{itemize}
				\item 3 families
				\item Period covered: march and April 2008
				\item Active all the time, several bursts of LFEs
			\end{itemize}
		\end{itemize}
	\end{frame}

	\begin{frame}
		\frametitle{Creating templates}
		% here prcedure
	\end{frame}

	\begin{frame}
		\frametitle{Creating templates}
		% here example of LFE template
	\end{frame}

	\begin{frame}
		\frametitle{Finding new LFEs}
		% here metod
	\end{frame}

	\begin{frame}
		\frametitle{Finding new LFEs}
		% here image of MAD and cc for an hour
	\end{frame}

	\begin{frame}
		\frametitle{Comparison with existing catalog}
		% here table of comparison
	\end{frame}

	\begin{frame}
		\frametitle{Extension of the catalog}
		% here family 080421.14.048
	\end{frame}
	
	\begin{frame}
		\frametitle{Extension of the catalog}
		% here family 0803.26.08.015
	\end{frame}

	\begin{frame}
		\frametitle{Detection of LFEs with permanent networks}
		% here visualization of two time series
	\end{frame}

	\begin{frame}
		\frametitle{Comparison FAME - permanent networks}
		% here cross correlation 080421.14.048
	\end{frame}

	\begin{frame}
		\frametitle{Comparison FAME - permanent networks}
		% here cross correlation 080326.08.015
	\end{frame}

	\begin{frame}
		\frametitle{Future work}
		\begin{itemize}
			\item Two-year-long catalog for all LFE families
			\item Computation of new templates for the permanent networks
			\item Whenever possible, extension of the LFE catalog to 2009-2019
		\end{itemize}
	\end{frame}

	\begin{frame}
		\frametitle{Effect of nearby earthquakes}
		% Example from San ndreas
	\end{frame}

	\begin{frame}
		\frametitle{Effect of nearby earthquakes}
		% here map of nearby earthquakes
	\end{frame}

	\begin{frame}
		\frametitle{Effect of nearby earthquakes}
		Future work:
		\begin{itemize}
			\item Event rate before the earthquake
			\item Event rate after the earthquake
			\item Comparison between two event rates: Computation of likelihood ratio
		\end{itemize}
	\end{frame}

	%----------------------------------------------------------------------------------------------------------------------------------
	% Statistical analysis of LFE catalogs
	%----------------------------------------------------------------------------------------------------------------------------------

	\subsection{Statistical analysis of LFE catalogs}

	\begin{frame}
		\frametitle{Homogeneous Poisson process}
		% Image
		Number of events recorded during one day $\rightarrow$ Variance over X values = Y
	\end{frame}

	\begin{frame}
		\frametitle{Homogeneous Poisson process}
		% Image
		Number of events recorded during 3 day $\rightarrow$ Variance over X values = Y
	\end{frame}

	\begin{frame}
		\frametitle{Homogeneous Poisson process}
		% Image
		Number of events recorded during 9 day $\rightarrow$ Variance over X values = Y
	\end{frame}

	\begin{frame}
		\frametitle{Homogeneous Poisson process}
		% Image
		Number of events recorded during 27 day $\rightarrow$ Variance over X values = Y
	\end{frame}

	\begin{frame}
		\frametitle{Homogeneous Poisson process}
		% Image
		Number of events recorded during 81 day $\rightarrow$ Variance over X values = Y
	\end{frame}

	\begin{frame}
		\frametitle{Homogeneous Poisson process}
		% Image
		Number of events recorded during 243 day $\rightarrow$ Variance over X values = Y
	\end{frame}

	\begin{frame}
		\frametitle{Variance vsLlength of time window}
		\begin{columns}[c]
			\begin{column}{5cm}
				$V$ = Variance
				$m$ = Sample size
				$V$ behaves as $m^{2 d - 1}$
				$\rightarrow$ $d =  0$
			\end{column}
			\begin{column}{7cm}
				\begin{center}
					% here figure
				\end{center}
			\end{column}
		\end{columns}
	\end{frame}

	\begin{frame}
		\frametitle{ETAS model}
		Epidemic-Type Aftershock Sequence (ETAS) model:
		\begin{itemize}
			\item Magnitude frequency distribution law of Gutemberg and Richter
			\item Omori-Utsu law of aftershock decay
			\item Each event, irrespective of whether it is a small or a big event, can trigger its own offspring
		\end{itemize}
		\begin{equation}
		\lambda = \mu + A \sum_{t_i < t} e^{\alpha \left( M_i - M_0 \right)} \left( 1 + \frac{t - t_i}{c} \right) ^{-p}
		\end{equation}
	\end{frame}

	\begin{frame}
		\frametitle{ETAS model}
		% Image
		Number of events recorded during 1 day $\rightarrow$ Variance over X values = Y
	\end{frame}

	\begin{frame}
		\frametitle{ETAS model}
		% Image
		Number of events recorded during 3 day $\rightarrow$ Variance over X values = Y
	\end{frame}

	\begin{frame}
		\frametitle{ETAS model}
		% Image
		Number of events recorded during 9 day $\rightarrow$ Variance over X values = Y
	\end{frame}

	\begin{frame}
		\frametitle{ETAS model}
		% Image
		Number of events recorded during 27 day $\rightarrow$ Variance over X values = Y
	\end{frame}

	\begin{frame}
		\frametitle{ETAS model}
		% Image
		Number of events recorded during 81 day $\rightarrow$ Variance over X values = Y
	\end{frame}

	\begin{frame}
		\frametitle{ETAS model}
		% Image
		Number of events recorded during 243 day $\rightarrow$ Variance over X values = Y
	\end{frame}

	\begin{frame}
		\frametitle{Variance vs Length of time window}
		\begin{columns}[c]
			\begin{column}{5cm}
				$V$ = Variance
				$m$ = Sample size
				$V$ behaves as $m^{2 d - 1}$
				$\rightarrow$ $d =  0.23$
			\end{column}
			\begin{column}{7cm}
				\begin{center}
					% here figure
				\end{center}
			\end{column}
		\end{columns}
	\end{frame}

	\begin{frame}
		\frametitle{Long-range dependence}
		\begin{itemize}
			\item Relates to the slow rate of decay of the statistical dependence between two points with increasing time interval between the points
			\item The fractional index $d$ represents how fast the variance in the number of LFEs in a time window of a given length increases with the length of the time window considered
			\item $0 < d < 0.5$ is characteristic of long-range dependence
		\end{itemize}
	\end{frame}

	\begin{frame}
		\frametitle{Northern Cascadia}
		% image
	\end{frame}

	\begin{frame}
		\frametitle{Guerrero, Mexico}
		% image
	\end{frame}

	\begin{frame}
		\frametitle{San Andreas Fault}
		% image
	\end{frame}

	\begin{frame}
		\frametitle{Future work}
		\begin{itemize}
			\item Computation of fractional index $d$ for the southern Cascadia catalog
			\item Fit ETAS model on existing LFE families
		\end{itemize}
	\end{frame}
		
	%----------------------------------------------------------------------------------------------------------------------------------
	% Detection of slow slip events in New Zealand
	%----------------------------------------------------------------------------------------------------------------------------------

	\section{Detection of slow slip events in New Zealand}

	\begin{frame}
		\frametitle{Tremor as proxy for slow slip}
		Episodic Tremor and Slip in Cascadia
		\begin{itemize}
			\item Tremor occurrence rate $\rightarrow$ Moment of slow slip events not detectable in the GPS data (Aguiar, 2009)
			\item Stacking of GPS data when LFEs are detected (Frank, 2016)
		\end{itemize}
		$\rightarrow$ What do we do when tremor is not correlated with slow slip?
	\end{frame}
		
	\begin{frame}
		\frametitle{Northern New Zealand}
		% Figure 1 of Wallace and Eerhardt Phillips 2010
	\end{frame}

	\begin{frame}
		\frametitle{Tremor during deep slow slip events}
		% Figure 3a of Wallace and Eerhardt Phillips 2010
	\end{frame}

	\begin{frame}
		\frametitle{Tremor during deep slow slip events}
		% Figure 3b of Wallace and Eberhardt Phillips 2010
	\end{frame}

	\begin{frame}
		\frametitle{Tremor with no detected slow slip event}
		% Figure 6 of Todd and Schartz 2016
	\end{frame}

	\begin{frame}
		\frametitle{Possible questions}
		\begin{itemize}
			\item 
			\item Detecting longer term slow slip events
			\item Better measuring of the vertical displacement at the Earth’s surface during slow slip events
		\end{itemize}
	\end{frame}

	\begin{frame}
		\frametitle{Detecting smaller, currently undetected slow slip events}
		% Figure 3 of Wech and Creager 2011
	\end{frame}

	\begin{frame}
		\frametitle{Detecting longer term slow slip events}
		% Figure 1 of Wei et al 2012
	\end{frame}

	\begin{frame}
		\frametitle{Better measuring of the vertical displacement at the Earth’s surface during slow slip events}
		% Figure from http://all-geo.org/highlyallochthonous/2015/05/the-geodetic-fingerprints-of-shallow-thrusting-in-nepal/
	\end{frame}

	\begin{frame}
		\frametitle{GPS stations in New Zealand}
		% map of statins
	\end{frame}

	\begin{frame}
		\frametitle{Denoising of GPS data}
		% Synthetics example
	\end{frame}

	\begin{frame}
		\frametitle{Denoising of GPS data}
		% Station CKID
	\end{frame}

	%----------------------------------------------------------------------------------------------------------------------------------
	% Time line
	%----------------------------------------------------------------------------------------------------------------------------------

	\section{Time line}

	\begin{frame}
		\begin{Huge}
			\begin{center}
				Questions?
			\end{center}
		\end{Huge}
	\end{frame}
			
\end{document}
