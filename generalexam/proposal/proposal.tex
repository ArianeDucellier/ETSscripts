\documentclass[letterpaper, 12pt]{article}

\usepackage{caption}
\usepackage[lmargin=1 in, rmargin=1 in, tmargin=1 in, bmargin=1 in]{geometry}
\usepackage{graphicx}
\usepackage{hyperref}
\usepackage{times}
\usepackage{xcolor}

\begin{document}

\setcounter{page}{1}

\begin{center}

\textbf{Data analysis of recordings of slow earthquakes: Tectonic tremor, low frequency earthquakes and slow slip events}

\vspace{1em}

Ariane Ducellier

PhD Dissertation Proposal

October 2019?

\end{center}

\section{Introduction}

\section{Proposed PhD Research}

\subsection{Depth of the source of the tectonic tremor in the eastern Olympic Peninsula}

\textit{Is the source of the tectonic tremor located on the plate boundary? What is the thickness of the tremor source region?}

\subsubsection*{Motivation}

\subsubsection*{Completed work}

\subsubsection*{Future work}

\subsection{A low-frequency earthquakes catalog for Northern California}

\textit{Do LFE families behave similarly or differently in Northern California, compared to northern Washington and the San Andreas fault?}

\subsubsection*{Motivation}

\subsubsection*{Completed work}

\subsubsection*{Future work}

\subsection{Detection of slow slip events in New Zealand}

\textit{Can we detect small and / or longer slow slip events in the absence of spatially and temporally correlated tectonic tremor?}

\subsubsection*{Motivation}

\subsubsection*{Completed work}

\subsubsection*{Future work}

\newpage
\setcounter{page}{1}

\bibliographystyle{plain}
\bibliography{bibliography}

\newpage
\setcounter{page}{1}

\section*{Figures}

\end{document}
